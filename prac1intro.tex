\documentclass{beamer}

\usepackage{listings}
\usepackage{color}
\usepackage{hyperref}

% Default fixed font does not support bold face
\DeclareFixedFont{\ttb}{T1}{txtt}{bx}{n}{10} % for bold
\DeclareFixedFont{\ttm}{T1}{txtt}{m}{n}{10}  % for normal

% Custom colors
\usepackage{color}
\definecolor{deepblue}{rgb}{0,0,0.5}
\definecolor{deepred}{rgb}{0.6,0,0}
\definecolor{deepgreen}{rgb}{0,0.5,0}

\usepackage{listings}

% Python style for highlighting
\newcommand\pythonstyle{\lstset{
language=Python,
basicstyle=\ttm,
otherkeywords={self},             % Add keywords here
keywordstyle=\ttb\color{deepblue},
emph={MyClass,__init__},          % Custom highlighting
emphstyle=\ttb\color{deepred},    % Custom highlighting style
stringstyle=\color{deepgreen},
frame=tb,                         % Any extra options here
showstringspaces=false            % 
}}


% Python environment
\lstnewenvironment{code}[1][]
{
%\begin{small}
\pythonstyle
\lstset{#1}
%\end{small}
}
{}


\begin{document}

% Python basics hands on lecture. Types and variables. if-then.

\begin{frame}
\frametitle{CS24420 \& MA25220 \& MT25220}
\begin{itemize}
\item CS24420 - Scientific Python
\item MA25220 - Introduction to Numerical Analysis and its
  applications
\item MT25220 - Cyflwyniad i Ddadansoddiad Rhifiadol a'i gymhwysiadau
\end{itemize}
\end{frame}

\begin{frame}[fragile]
\frametitle{Open Spyder}
There are 3 windows: editor, help and console.

You can see the console shown is an iPython console. It uses an
\texttt{In [1]:} prompt.

There is also a plain console that you can switch to. It uses a \texttt{>>>} prompt.

\end{frame}


\begin{frame}[fragile]
\frametitle{Try the following}
Open Spyder and try the following in the console window:
\begin{code}
>>> 1 + 5
>>> 6 * 3
>>> 7 / 3
\end{code}

Also try it in the iPython console window:
\begin{code}
In [1]: 1 + 5
In [2]: 6 * 3
In [3]: 7 / 3
\end{code}

\end{frame}



\begin{frame}[fragile]
\frametitle{Try the following}
\begin{code}
>>> 2 + 3 * 7
>>> (2 + 3) * 7
>>> 2 + (3 * 7)
\end{code}
\end{frame}


\begin{frame}[fragile]
\frametitle{Save to a file}
Make a folder in your m: drive to save your work for this module.

Type the following code into the editor pane. 

Save the code to a file in your m: drive called hello.py. It must have
a .py extension. 

Then use the green triangle ('Run File') button to run your code.
\begin{code}
# This is a comment line that the computer will ignore
x = 1 + 5
y = 6 / 4.0
print("Hello World")

# Lets see what x and y contain
print(x)
print(y)
\end{code}
\end{frame}

\begin{frame}[fragile]
\frametitle{Check numpy and matplotlib work}
Check that numpy and matplotlib are available and work for you:
\begin{code}
import numpy as np
import matplotlib.pyplot as plt

print("Hello World")
print(np.zeros(3))
plt.plot([2,3,4]) 
\end{code}
\end{frame}

\end{document}
