\documentclass{beamer}

\usepackage{listings}
\usepackage{color}
\usepackage{hyperref}

\usepackage[T1]{fontenc}
\usepackage{textcomp}
\usepackage{upquote}

% Default fixed font does not support bold face
\DeclareFixedFont{\ttb}{T1}{txtt}{bx}{n}{10} % for bold
\DeclareFixedFont{\ttm}{T1}{txtt}{m}{n}{10}  % for normal

% Custom colors
\usepackage{color}
\definecolor{deepblue}{rgb}{0,0,0.5}
\definecolor{deepred}{rgb}{0.6,0,0}
\definecolor{deepgreen}{rgb}{0,0.5,0}
\definecolor{shadecolor}{rgb}{1, 0.9, 0.3}

\usepackage{listings}

% Python style for highlighting
\newcommand\pythonstyle{\lstset{
language=Python,
basicstyle=\ttm,
otherkeywords={self},             % Add keywords here
keywordstyle=\ttb\color{deepblue},
emph={MyClass,__init__},          % Custom highlighting
emphstyle=\ttb\color{deepred},    % Custom highlighting style
stringstyle=\color{deepgreen},
frame=tb,                         % Any extra options here
showstringspaces=false,            % 
upquote=True,
columns=fullflexible,
basicstyle=\ttfamily
}}


% Python environment
\lstnewenvironment{code}[1][]
{
%\begin{small}
\pythonstyle
\lstset{#1}
%\end{small}
}
{}


\begin{document}

% Theory of objects. (maybe Inheritance. Design). Objects = methods plus attributes. Self. Calling methods via dot vs just calling functions.  __init__. private.

\begin{frame}
\frametitle{CS24420 \& MA25220 \& MT25220 \& MX35220 \& CSM0120}

\begin{center}
\begin{huge}
Lecture 13: Object-oriented code 
\end{huge}
\bigskip

Alexander Pitchford (agp1@aber.ac.uk)

\end{center}
\end{frame}

%---------------------------------------------------
\begin{frame}[fragile]
\frametitle{A World Full of Objects}
The world is full of objects:
\begin{itemize}
\item chairs
\item tables
\item buildings
\item planets
\item stars
\item washing machines
\end{itemize}

\end{frame}

%----------------------------------------------------
\begin{frame}[fragile]
\frametitle{Things}
The world also contains other things
%SpeakerNote: Things that we would not usually call objects
\begin{itemize}
\item atoms
\item animals
\item people
\item bacteria
\item emotions
\item philosophical concepts
\end{itemize}

\bigskip
All these objects and things have characteristics\\
% chairs have a number of legs
% plants have a mass and a radius
% washing machines have a model name
\smallskip
and possibly actions associated with them
% either things they do or have done to them
% stars radiate
% washing machines wash clothes
% emotions can expressed
% philosophical concepts can be argued
\bigskip

\end{frame}

%----------------------------------------------------
\begin{frame}[fragile]
\frametitle{Object-oriented Programming}

\begin{itemize}
\item Naturally then we would like some way of representing these things in code
\item In many computer programming languages objects and things can be represented by \emph{objects}
\item Characteristics by \emph{attributes} or \emph{properties}
\item Actions by \emph{methods}
\item Languages that support this are called \emph{Object-Oriented}
\item or 'OO' for short
\end{itemize}

\end{frame}

%----------------------------------------------------
\begin{frame}[fragile]
\frametitle{Objects in Python}
\begin{itemize}
\item Python is an Object-Oriented language
% Not as OO as some others, like C++ or Java
% Many people happily write Python using module and functions 
% and rarely, if ever, write a custom class
\item Objects are everywhere in Python
\item In all the workshops we have used objects
\item All of these are objects:
\begin{itemize}
\item strings
\item files
\item modules
\item lists, dictionaries, numpy arrays
\item Exceptions
\item in fact all variables of all types
\end{itemize}
\end{itemize}
% Switch to live demo:
% showing how strings have methods like upper() and lower()
% numpy has a __name__ attribute
% an Exception has a string representation
% numbers have a real and imaginary part
\begin{code}
"ABC".lower()
numpy.__name__
\end{code}
\end{frame}

%----------------------------------------------------
\begin{frame}[fragile]
\frametitle{Creating an Object}
\begin{itemize}
\item So we 'accidentally' create objects all the time whenever we set a variable value
\item However, we will focus here on objects from modules
\item These are examples of creating on object.
\end{itemize}

\begin{code}
myobj = ClassName()
mach1 = WashingMachine()
pfact = PeopleFactory()
\end{code}

\end{frame}

%----------------------------------------------------
\begin{frame}[fragile]
\frametitle{Creating an Object - with arguments}

\begin{itemize}
\item In the previous examples no arguments are given
\item meaning that a generic instance of the object will be created
\item Sometimes we create them with \emph{initialisation arguments}
\item These generate objects with some attributes already set to specific values
\end{itemize}

\begin{code}
myobj = ClassName(init_arg1, init_arg2)
mach1 = WashingMachine("Hotpoint WMFUG842G")
sam = Person("Sam Nicholls", 'male', 25, 1.6)
pstats = PersonalStatistics(meandmyfriends)
\end{code}
\end{frame}

%----------------------------------------------------
\begin{frame}[fragile]
\frametitle{Using Objects}
bar
\begin{code}
x = 1 + 5
\end{code}
\end{frame}

%----------------------------------------------------
\begin{frame}[fragile]
\frametitle{Defining Objects - Classes}
% Naming convention 'PascalCase'
\begin{code}
x = 1 + 5
\end{code}
\end{frame}

%----------------------------------------------------
\begin{frame}[fragile]
\frametitle{Class Initialisation}
bar
\begin{code}
x = 1 + 5
\end{code}
\end{frame}

%----------------------------------------------------
\begin{frame}[fragile]
\frametitle{Attributes}
bar
\begin{code}
x = 1 + 5
\end{code}
\end{frame}

%----------------------------------------------------
\begin{frame}[fragile]
\frametitle{Methods}
bar
\begin{code}
x = 1 + 5
\end{code}
\end{frame}

%----------------------------------------------------
\begin{frame}[fragile]
\frametitle{Example - WashingMachine}
bar
\begin{code}
x = 1 + 5
\end{code}
\end{frame}


%----------------------------------------------------
\begin{frame}[fragile]
\frametitle{Classification Systems}
Tree of Life
Standard model
\begin{code}
x = 1 + 5
\end{code}
\end{frame}

%----------------------------------------------------
\begin{frame}[fragile]
\frametitle{Inheritance}
bar
\begin{code}
x = 1 + 5
\end{code}
\end{frame}

%----------------------------------------------------
\begin{frame}[fragile]
\frametitle{Example - WasherDryer}
bar
\begin{code}
x = 1 + 5
\end{code}
\end{frame}

%----------------------------------------------------
\begin{frame}[fragile]
\frametitle{When to use Custom Objects}
bar
\begin{code}
x = 1 + 5
\end{code}
\end{frame}

%----------------------------------------------------
\begin{frame}[fragile]
\frametitle{Private attributes and Methods}
bar
\begin{code}
x = 1 + 5
\end{code}
\end{frame}

%----------------------------------------------------
\begin{frame}[fragile]
\frametitle{Python 'system' attributes and Methods}
bar
\begin{code}
x = 1 + 5
\end{code}
\end{frame}

\end{document}