\documentclass{beamer}

\usepackage{listings}
\usepackage{color}
\usepackage{hyperref}

\usepackage[T1]{fontenc}
\usepackage{textcomp}

% Default fixed font does not support bold face
\DeclareFixedFont{\ttb}{T1}{txtt}{bx}{n}{10} % for bold
\DeclareFixedFont{\ttm}{T1}{txtt}{m}{n}{10}  % for normal

% Custom colors
\usepackage{color}
\definecolor{deepblue}{rgb}{0,0,0.5}
\definecolor{deepred}{rgb}{0.6,0,0}
\definecolor{deepgreen}{rgb}{0,0.5,0}
\definecolor{purple}{RGB}{153, 0, 153}

\usepackage{listings}

% Python style for highlighting
\newcommand\pythonstyle{\lstset{
language=Python,
basicstyle=\ttm,
otherkeywords={self},             % Add keywords here
keywordstyle=\ttb\color{deepblue},
emph={MyClass,__init__},          % Custom highlighting
emphstyle=\ttb\color{deepred},    % Custom highlighting style
stringstyle=\color{deepgreen},
frame=tb,                         % Any extra options here
showstringspaces=false,            % 
upquote=True,
columns=fullflexible,
basicstyle=\ttfamily
}}


% Python environment
\lstnewenvironment{code}[1][]
{
%\begin{small}
\pythonstyle
\lstset{#1}
%\end{small}
}
{}


\begin{document}

% numpy hands on lecture (make arrays, apply scalars, make ndarrays, apply basic functions).

\begin{frame}
\frametitle{CS24420 \& MA25220 \& MT25220 \& MX35220 \& CSM0120}

\begin{center}
\begin{huge}
Lecture 6: Hands-on NumPy
\end{huge}
\bigskip

Amanda Clare (afc@aber.ac.uk)

\end{center}
\end{frame}


\begin{frame}[fragile]
\frametitle{Multidimensional arrays}
We've seen lists, dictionaries and sets. The numpy module gives us
multidimensional arrays ('ndarray' or 'array').  

\begin{itemize}
\item elements all of the same type
\item the dimensions are called axes
\item the number of dimensions is the rank 
\end{itemize}

\bigskip

Today: hands on
experimenting, and Tuesday will cover the theory.

\bigskip

The numpy module is often imported as follows to save characters later:
\begin{code}
import numpy as np
\end{code}

\end{frame}



\begin{frame}[fragile]
\frametitle{Multidimensional arrays}
We can make arrays of different shapes containing random numbers:
\begin{code}
>>> import numpy as np
>>> x = np.random.random(10)
>>> print(x)
>>> y = np.random.random((3, 10))
>>> print(y)
>>> y = np.random.random((10, 3))
>>> print(y)
>>> y = np.random.random((3, 3, 3))
>>> print(y)
\end{code}
\end{frame}

\begin{frame}[fragile]
\frametitle{Multidimensional arrays}
Inspect the array. 
\begin{code}
>>> y = np.random.random((10, 3))
>>> print(y)
>>> y.shape
>>> y.dtype
>>> type(y)
\end{code}
\end{frame}

\begin{frame}[fragile]
\frametitle{Other ways to make arrays}
\begin{itemize}
\item a = np.zeros(5)
\pause
\item b = np.ones(5)
\pause
\item c = 5 * np.ones(5)
\pause
\item d = np.array([[5,5,5,5,5], [1,1,1,1,1], [8,8,8,8,8]])
\pause 
\item e = np.array([b * 5, b, b * 8])
\pause
\item f = np.array([b * 5, b, b * 8]).astype(int)
\end{itemize}
\end{frame}

\begin{frame}[fragile]
\frametitle{Slicing}
Slice the array. What kind of slices are possible?
\begin{code}
>>> y = np.random.random((10, 3))
>>> print(y)
>>> print(y[5, 2])   # element
>>> print(y[5])    # row
>>> print(y[5, :])    # row
>>> print(y[:, 2])    # col
>>> print(y[4:6, 1:3])   # 2d slice
\end{code}
\end{frame}



\begin{frame}[fragile]
\frametitle{Watch out when using slices}
A slice is not a copy of an ndarray. Instead it's just another view.

\bigskip

If z is a slice of y and then the
values in z are modified, this will modify the original y too.

\begin{code}
>>> y = np.random.random((10, 3))
>>> print(y)
>>> z = y[4:6, 1:3]
>>> z[0, 0] = 0
>>> print(z)
>>> print(y)
\end{code}
\end{frame}


\begin{frame}[fragile]
\frametitle{Watch out when using slices}
What's the difference between these two ways to request the first row?
\begin{code}
>>> y = np.random.random((10, 3))
>>> print(y)
>>> print(y[0])
>>> print(y[0, :])
>>> print(y[0:1, :])
\end{code}
\end{frame}

\begin{frame}[fragile]
\frametitle{Scalar operations}
np.random.random will generate numbers between 0 and 1 (not including
1). What if we want random numbers between 0 and 100?
\begin{code}
>>> x = np.random.random(10) * 100
>>> print(x)
>>> y = np.random.random((3, 10)) * 100
>>> print(y)
\end{code}
\end{frame}

\begin{frame}[fragile]
\frametitle{Scalar operations}
What's the difference? Why?
\begin{code}
>>> x = np.random.random((3, 5)) * 10 + 5
>>> print(x)
>>> y = np.random.random((3, 5)) + 5 * 10
>>> print(y)
\end{code}
\end{frame}

\begin{frame}[fragile]
\frametitle{Converting Celsuis to Fahrenheit}
\begin{code}
>>> temps = np.array([15, 18, 23, 20])
>>> f_temps = temps * 9/5 + 32
>>> print(f_temps)
\end{code}
\end{frame}



\begin{frame}[fragile]
\frametitle{Boolean indexing of arrays}
\begin{code}
>>> x = np.array([[1,2,3],[4,5,6],[7,8,9]])
>>> x
array([[1, 2, 3],
       [4, 5, 6],
       [7, 8, 9]])
>>> x < 5
array([[ True,  True,  True],
       [ True, False, False],
       [False, False, False]], dtype=bool)
\end{code}
\end{frame}

\begin{frame}[fragile]
\frametitle{Boolean indexing of arrays}
We can use a boolean array (such as x < 5) as an index to request
certain values in x.
\begin{code}
>>> x[x < 5]

>>> x[x % 2 == 0]

>>> x[x ** 2 < 50]
\end{code}
\end{frame}

\begin{frame}[fragile]
\frametitle{Boolean indexing of arrays}
We can use a boolean array (such as x < 5) as an index into another
array (such as y).
\begin{code}
>>> x = np.array([[2,2],[4,4],[3,5]])
>>> y = np.array([[1,1],[1,1],[1,1]])
>>> y[x < 4]
>>> print(y)
\end{code}

And we can use the boolean array as a mask to choose which values to
operate on.
\begin{code}
>>> y[x < 4] = 0
>>> print(y)
\end{code}
\end{frame}


\begin{frame}[fragile]
\frametitle{Maximum values}
We can find the maximum value in each row or column using \texttt{amax}
(maximum along an axis, where 0 = columns, 1 = rows). Also
try \texttt{amin}, \texttt{mean}, \texttt{std}, \texttt{sum}.
\begin{code}
>>> x = np.random.random((10,5))
>>> np.amax(x, 0)
>>> np.amax(x, 1)
>>> np.amax(x)
\end{code}
\end{frame}

\begin{frame}[fragile]
\frametitle{Mars}
Heightmap data for the surface of Mars from
\url{http://pds-geosciences.wustl.edu/missions/mgs/megdr.html} (meg004,
topography).
\pause
\begin{code}
# Read the data from the file as a 
# 1-dimensional array of 16-bit ints
a = np.fromfile("megt90n000cb.img", dtype=np.uint16)

# Reshape the array to have rows and columns
b = a.reshape(720, 1440)
print(b)
\end{code}
See the extra code provided in lec6.py, which also includes code to plot the
results. A zipfile with the code and data is available on Blackboard. 
\end{frame}

\begin{frame}[fragile]
\frametitle{Summary}
\begin{itemize}
\item import numpy as np
\item n-dimensional arrays
\item slicing
\item scalar operations
\item boolean indexing
\item applying functions to an axis 
\item reading and plotting an array as an image
\end{itemize}
\end{frame}

\end{document}
