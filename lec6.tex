\documentclass{beamer}

\usepackage{listings}
\usepackage{color}
\usepackage{hyperref}

% Default fixed font does not support bold face
\DeclareFixedFont{\ttb}{T1}{txtt}{bx}{n}{10} % for bold
\DeclareFixedFont{\ttm}{T1}{txtt}{m}{n}{10}  % for normal

% Custom colors
\usepackage{color}
\definecolor{deepblue}{rgb}{0,0,0.5}
\definecolor{deepred}{rgb}{0.6,0,0}
\definecolor{deepgreen}{rgb}{0,0.5,0}

\usepackage{listings}

% Python style for highlighting
\newcommand\pythonstyle{\lstset{
language=Python,
basicstyle=\ttm,
otherkeywords={self},             % Add keywords here
keywordstyle=\ttb\color{deepblue},
emph={MyClass,__init__},          % Custom highlighting
emphstyle=\ttb\color{deepred},    % Custom highlighting style
stringstyle=\color{deepgreen},
frame=tb,                         % Any extra options here
showstringspaces=false            % 
}}


% Python environment
\lstnewenvironment{code}[1][]
{
%\begin{small}
\pythonstyle
\lstset{#1}
%\end{small}
}
{}


\begin{document}

% numpy hands on lecture (make arrays, apply scalars, make ndarrays, apply basic functions).

\begin{frame}
\frametitle{CS24420 \& MA25220 \& MT25220}

\begin{center}
\begin{huge}
Lecture 6: Hands-on NumPy
\end{huge}
\bigskip

Amanda Clare (afc@aber.ac.uk)

\end{center}
\end{frame}


\begin{frame}[fragile]
\frametitle{Multidimensional arrays}
\begin{itemize}
\item elements all of the same type
\item the dimensions are called axes
\item the number of dimensions is the rank 
\end{itemize}

\end{frame}

\begin{frame}[fragile]
\frametitle{Multidimensional arrays}
\begin{code}
>>> import numpy as np
>>> x = np.random.random(10)
>>> print(x)
>>> y = np.random.random((3, 10))
>>> print(y)
>>> y = np.random.random((10, 3))
>>> print(y)
>>> y = np.random.random((3, 3, 3))
>>> print(y)
\end{code}
\end{frame}

\begin{frame}[fragile]
\frametitle{Scalar operations}

np.random.random will generate numbers between 0 and 1 (not including
1). What if we want numbers between 0 and 100?
\begin{code}
>>> x = np.random.random(10) * 100
>>> print(x)
>>> y = np.random.random((3, 10)) * 100
>>> print(y)
\end{code}
\end{frame}

\begin{frame}[fragile]
\frametitle{Scalar operations}

What's the difference? Why?
\begin{code}
>>> x = np.random.random(3, 5) * 10 + 5
>>> print(x)
>>> y = np.random.random((3, 5)) + 5 * 10
>>> print(y)
\end{code}
\end{frame}



\end{document}
