\documentclass{beamer}

\usepackage{listings}
\usepackage{color}
\usepackage{hyperref}

\usepackage[T1]{fontenc}
\usepackage{textcomp}
\usepackage{upquote}

% Default fixed font does not support bold face
\DeclareFixedFont{\ttb}{T1}{txtt}{bx}{n}{10} % for bold
\DeclareFixedFont{\ttm}{T1}{txtt}{m}{n}{10}  % for normal

% Custom colors
\usepackage{color}
\definecolor{deepblue}{rgb}{0,0,0.5}
\definecolor{deepred}{rgb}{0.6,0,0}
\definecolor{deepgreen}{rgb}{0,0.5,0}
\definecolor{shadecolor}{rgb}{1, 0.9, 0.3}

\usepackage{listings}

% Python style for highlighting
\newcommand\pythonstyle{\lstset{
language=Python,
basicstyle=\ttm,
otherkeywords={self},             % Add keywords here
keywordstyle=\ttb\color{deepblue},
emph={MyClass,__init__},          % Custom highlighting
emphstyle=\ttb\color{deepred},    % Custom highlighting style
stringstyle=\color{deepgreen},
frame=tb,                         % Any extra options here
showstringspaces=false,            % 
upquote=True,
columns=fullflexible,
basicstyle=\ttfamily
}}


% Python environment
\lstnewenvironment{code}[1][]
{
%\begin{small}
\pythonstyle
\lstset{#1}
%\end{small}
}
{}


\begin{document}

% Theory of exceptions. What they are, how you catch them, how to continue. Why we have them. Exceptions vs if-then. What's caught at run time and what's caught before. 

\begin{frame}
\frametitle{CS24420 \& MA25220 \& MT25220 \& MX35220 \& CSM0120}

\begin{center}
\begin{huge}
Lecture 11: A tour of Python's exceptions
\end{huge}
\bigskip

Amanda Clare (afc@aber.ac.uk)

\end{center}
\end{frame}

\begin{frame}[fragile]
\frametitle{Worksheets}
\begin{enumerate}
\item Familiarisation
\item Variables, if and while
\item Lists, dictionaries and sets (also using for-loops, input from
  keyboard).
\item Numpy (sign-off today)
\pause
\item Starting today: Functions and documentation
\end{enumerate}
\pause
Each score 0, 1, or 2.

There will be three more worksheets (6, 7, 8). Worksheet 8 will be
shorter (and will therefore be scored 0 or 1). 

Total score possible = 15 (which is 7 worksheets at 2 points + 1
worksheet at 1 point). 
\end{frame}

\begin{frame}[fragile]
\frametitle{Next week is work week}
\begin{itemize}
\item No new material (no lectures).
\item Time to catch up on worksheets and revision.
\item There will be a practical on Tues 15th.
\item There will be a practical on Friday 18th (optional).
\item There will be no lecture on Tues 15th.
\end{itemize}
\end{frame}

\begin{frame}[fragile]
\frametitle{Errors when code is run}
\begin{code}
>>> 6 / 0

>>> x = 5
>>> y = 9 - (14 - x)
>>> x / y

>>> 6 % 0
\end{code}
\end{frame}

\begin{frame}[fragile]
\frametitle{ZeroDivisionError}
\begin{code}
>>> x = 5
>>> y = 9 - (14 - x)
>>> x / y
Traceback (most recent call last):
File "<stdin>", line 1, in <module>
ZeroDivisionError: integer division or modulo by zero
\end{code}

\bigskip

\textbf{ZeroDivisionError} is an exception that is \textbf{thrown} by your
code.
\end{frame}

\begin{frame}[fragile]
\frametitle{IndexError}
\begin{code}
>>> x = [4,7,3,8]
>>> x[6]
Traceback (most recent call last):
File "<stdin>", line 1, in <module>
IndexError: list index out of range
\end{code}
\textbf{IndexError} is another possible exception.
\end{frame}

\begin{frame}[fragile]
\frametitle{More exceptions}
\begin{itemize}
\item ZeroDivisionError
\item IndexError
\item KeyError
\item IOError
\item FileNotFoundError
\item ValueError (e.g. math.sqrt(-1.0))
\end{itemize}
\begin{itemize}
\item ImportError
\item NameError
\item SyntaxError
\item IndentationError
\item TypeError
\end{itemize}
\begin{itemize}
\item numpy.LinAlgError
\end{itemize}
\end{frame}

\begin{frame}[fragile]
\frametitle{How to handle the exceptions}
\begin{code}
x = 5
y = 9 - (14 - x)
try:
   z = x / y
except ZeroDivisionError:
   z = 54
print(z)
\end{code}
\end{frame}

\begin{frame}[fragile]
\frametitle{How to handle the exceptions}

\begin{code}
while True:
   try:
      x = int(input("Please enter a number: "))
      break
   except ValueError:
      print("That was not a valid number.")
      print("Please enter a number.")
\end{code}

\bigskip

Note: \texttt{break} allows us to immediately exit the while loop and move
to the next line of code following the while loop.
\end{frame}

\begin{frame}[fragile]
\frametitle{How to handle the exceptions}
\begin{code}
x = [3,7,5,2,9]
try:
   tenth_elem = x[9]
except IndexError:
   tenth_elem = -1
print(tenth)
\end{code}
\end{frame}

\begin{frame}[fragile]
\frametitle{What's the problem?}
\begin{code}
x = [3,7,5,2,9]
try:
   tenth = x[9]
except IndexError:
   print("Sorry, not enough samples in the data")
   print("There are only " + str(len(x)))
print(tenth)
\end{code}
\end{frame}

\begin{frame}[fragile]
\frametitle{Could quit the program}
\begin{code}
import sys
x = [3,7,5,2,9]
try:
   tenth = x[9]
except IndexError:
   print("Sorry, not enough samples in the data")
   print("There are only " + str(len(x)))
   sys.exit(1)
print(tenth)
\end{code}
\end{frame}

\begin{frame}[fragile]
\frametitle{Could continue with else}
\begin{code}
x = [3,7,5,2,9]
try:
   tenth = x[9]
except IndexError:
   print("Sorry, not enough samples in the data")
   print("There are only " + str(len(x)))
else:
   print(tenth)
\end{code}
\end{frame}

\begin{frame}[fragile]
\frametitle{Catch a more generic error}
(This is not advised, you should always try to catch more specific errors)
\begin{code}
x = [3,7,5,2,9]
try:
   tenth = x[9]
except Exception:
   print("Something went wrong")
else:
   print(tenth)
\end{code}
\end{frame}

\begin{frame}[fragile]
\frametitle{Print info about the exception}
\begin{code}
x = [3,7,5,2,9]
try:
   tenth = x[9]
except Exception as e:
   print("There was an exception and the arguments were " 
            + str(e.args))
else:
   print(tenth)
\end{code}
The args are a tuple. The tuple may contain anything.
\end{frame}

\begin{frame}[fragile]
\frametitle{Problems when catching generic errors}
\begin{code}
x = [3,7,5,2,9]
try:
   tenth = y[9]
except Exception:
   print("Element 9 doesn't exist")
else:
   print(tenth)
\end{code}
This will print ``Element 9 doesn't exist''. If instead we had
caught IndexError, then the code would have terminated with a
NameError to tell us that ``name 'y' is not defined''.
\end{frame}

\begin{frame}[fragile]
\frametitle{Exceptions vs if-else}
The word 'exception' makes us think that these are exceptional
events (events that occur very rarely).

Actually they may occur frequently. You might use try-except regularly
in your code.

You might even use try-except instead of if-else. You might do this for two
reasons:
\begin{itemize}
\item faster code
\item cleaner looking code
\end{itemize}
This is not true for most programming languages. However because
Python is a dynamic language, many more things are decided at run
time. So this is a Pythonic approach.
\end{frame}

\begin{frame}[fragile]
\frametitle{Exceptions vs if-else}
Faster code:
\begin{code}
if len(forest_means) > 100:
   hundredth = forest_means[100]
else:
   hundredth = None
\end{code}

\bigskip

\begin{code}
try:
   hundredth = forest_means[100]
except IndexError:
   hundredth = None
\end{code}
\end{frame}


\begin{frame}[fragile]
\frametitle{Exceptions vs if-else}
Cleaner code?
\begin{code}
word = sys.stdin.readline().strip()
if isinstance(word,str)  and word.isdigit() and len(word)<10:
   n = int(word)
else:
   n = None
\end{code}

\bigskip

compare with:

\begin{code}
word = sys.stdin.readline().strip()
try:
    return int(word)
except (TypeError, ValueError, OverflowError): 
    return None
\end{code}
``Easier to Ask Forgiveness than Permission'' 
\end{frame}

\begin{frame}[fragile]
\frametitle{Raise your own exceptions}
\begin{code}
def get_age():
   print("Enter your age") 
   age = int(sys.stdin.readline())
   if age < 0:
      raise ValueError("%d is not a valid age" % age)
   return age
\end{code}
\end{frame}


\begin{frame}[fragile]
\frametitle{}
\begin{code}
>>> x = test.get_age()
Enter your age
-3
Traceback (most recent call last):
  File "<stdin>", line 1, in <module>
  File "test.py", line 7, in get_age
    raise ValueError("%d is not a valid age" % age)
ValueError: -3 is not a valid age
\end{code}
\end{frame}


\begin{frame}[fragile]
\frametitle{Turtle}
And now for something completely different.
\end{frame}


\begin{frame}[fragile]
\frametitle{Turtle}
Turtle is a library for drawing turtle graphics. It's great for
teaching in schools. For example:

\begin{code}
import turtle

t = turtle.Turtle()
t.forward(50)
t.left(90)
t.forward(50)
\end{code}

\bigskip

Now we're going to look at a few more examples: lec11a.py and
lec11b.py
\end{frame}

\begin{frame}
\frametitle{Summary}
Exceptions:
\begin{itemize}
\item what they are
\item how to handle them
\item how to raise them
\end{itemize}

Turtle:
\begin{itemize}
\item Use it to practice loops, variables, functions, etc.
\item Take it into schools and encourage the next generation
\end{itemize}
\end{frame}

\end{document}
