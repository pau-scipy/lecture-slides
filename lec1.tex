\documentclass{beamer}

\usepackage{listings}
\usepackage{color}
\usepackage{hyperref}



\begin{document}

\begin{frame}
\frametitle{CS24420 \& MA25220 \& MT25220}

\begin{center}
\begin{huge}
Lecture 1: Introduction 
\end{huge}
\bigskip

Amanda Clare (afc@aber.ac.uk)

\end{center}
\end{frame}

\begin{frame}
\frametitle{CS24420 \& MA25220 \& MT25220}
\begin{itemize}
\item CS24420 - Scientific Python
\item MA25220 - Introduction to Numerical Analysis and its
  applications
\item MT25220 - Cyflwyniad i Ddadansoddiad Rhifiadol a'i gymhwysiadau
\item CSM0120 - Python for Scientists
\end{itemize}
\end{frame}

\begin{frame}
\frametitle{CS24420 \& MA25220 \& MT25220}
\begin{itemize}
\item Module content on Blackboard.
\item 2 lectures and a practical each week (all compulsory).
\begin{itemize}
\item Tues 4:10pm A6 - lecture
\item Tues 5:10pm B23 - prac
\item Friday 13:10pm B23 - lecture 
\end{itemize}
\item Module coordinators: Amanda Clare (afc), Tudur Davies (itd)
\end{itemize}
\end{frame}

\begin{frame}
\frametitle{CS24420 \& MA25220 \& MT25220}
Different assessments:
\begin{itemize}
\item CS24420: 30\% worksheets, 70\% exam
\item MA25220/MT25220: 30\% worksheets, 20\% assignments, 50\% exam
\item CSM0120: 2 assignments (40\%, 60\%) 
\end{itemize}
\end{frame}

\begin{frame}
\frametitle{Worksheets}
\begin{itemize}
\item One every week, including today
\item Graded 0, 1, 2
\item Signed off by demonstrators
\item You have a maximum of 3 practical sessions (15 days) for each worksheet, before it gets a 0
\item For example you must get today's worksheet signed off by the end
  of next Tues at the latest.
\item Any problems (sickness etc), see me.
\end{itemize}
\end{frame}


\begin{frame}
\frametitle{Books}
\begin{itemize}
\item Automate the Boring Stuff with Python, Al Sweigart 
    \url{https://automatetheboringstuff.com/\#toc}
\item Many other Python books in library and in bookshops
\item See recommendations on Blackboard
\end{itemize}
\end{frame}

\begin{frame}
\frametitle{Python creation}
\begin{itemize}
\item First created in 1989 by Guido van Rossum
\item Named after comedy TV show Monty Python's Flying Circus
\item Owned by the Python Software Foundation (PSF), a non-profit org
\item van Rossum is Benevolent Dictator For Life
\item van Rossum has worked for Google, now works for Dropbox
\end{itemize}
\end{frame}



\begin{frame}
\frametitle{Topics for Semester 1}
\begin{itemize}
\item Language basics
\item Data structures
\item NumPy
\item Functions
\item Organising code
\item Objects
\item File handling
\item Plotting
\end{itemize}
\end{frame}

\begin{frame}
\frametitle{Topics for Semester 2}
\begin{itemize}
\item CS24420 - Science, noise, statistics, hot topics
\item MA25220/MT25220 - Numerical approximations to maths problems
\end{itemize}
\end{frame}




\begin{frame}
\frametitle{Python 2 vs Python 3}
\begin{itemize}
\item There are two versions of Python in active use in industry.
\item Although Python 3 has been around for many years, legacy code
  has been slow to move.
\item This course will mention the differences as we come across them.
\end{itemize}
\end{frame}

\begin{frame}
\frametitle{Style of Python}
\begin{itemize}
\item Scripting, OO, functional, imperative.
\item Get things done quickly.
\item Many libraries.
\item Large community, lots of cheap books and example code.
\item Uses whitespace to indicate code structure
\end{itemize}
\end{frame}

\begin{frame}
\frametitle{Anaconda, libraries and scientific python}
\begin{itemize}
\item Many libraries are bundled with Python
\item Some scientific libraries are not
\item To easily install Python with scientific libraries on Mac or Windows, try the
  Anaconda distribution (\url{https://www.continuum.io/downloads}
\end{itemize}
\end{frame}

\begin{frame}
\frametitle{Editors, IDES, consoles}
\begin{itemize}
\item gedit/aquamacs/vim/emacs/notepad++ 
\item Spyder IDE
\item Interactive console vs iPython vs coding in files/modules
\item No compiler, Python is an interpreted language
\end{itemize}
\end{frame}


\begin{frame}
\frametitle{Consoles}
\begin{itemize}
\item Python console 
\item iPython notebook
\item Demo of the environment you'll have in B23
\end{itemize}
\end{frame}


\begin{frame}
\frametitle{Spyder}
\begin{itemize}
\item Pros and cons of an IDE vs editor plus console
\item Try it and see what you think
\end{itemize}
\end{frame}



\begin{frame}
\frametitle{Let's get installed}
\begin{itemize}
\item Off to B23 to ensure your uni setup is good 
\item And install it on your laptops if wanted.
\end{itemize}
\end{frame}


\end{document}
