\documentclass{beamer}

\usepackage{listings}
\usepackage{color}
\usepackage{hyperref}

\usepackage[T1]{fontenc}
\usepackage{textcomp}
\usepackage{upquote}

% Default fixed font does not support bold face
\DeclareFixedFont{\ttb}{T1}{txtt}{bx}{n}{10} % for bold
\DeclareFixedFont{\ttm}{T1}{txtt}{m}{n}{10}  % for normal

% Custom colors
\usepackage{color}
\definecolor{deepblue}{rgb}{0,0,0.5}
\definecolor{deepred}{rgb}{0.6,0,0}
\definecolor{deepgreen}{rgb}{0,0.5,0}
\definecolor{shadecolor}{rgb}{1, 0.9, 0.3}


\usepackage{listings}

% Python style for highlighting
\newcommand\pythonstyle{\lstset{
language=Python,
basicstyle=\ttm,
otherkeywords={self},             % Add keywords here
keywordstyle=\ttb\color{deepblue},
emph={MyClass,__init__},          % Custom highlighting
emphstyle=\ttb\color{deepred},    % Custom highlighting style
stringstyle=\color{deepgreen},
frame=tb,                         % Any extra options here
showstringspaces=false,            % 
upquote=True,
columns=fullflexible,
basicstyle=\ttfamily
}}


% Python environment
\lstnewenvironment{code}[1][]
{
%\begin{small}
\pythonstyle
\lstset{#1}
%\end{small}
}
{}


\begin{document}

% Hands on adding main to a file of functions. Importing your
% file. Running your file. Adding comments. Generating documentation
% for your module. Packages, pip, argparse


\begin{frame}
\frametitle{CS24420 \& MA25220 \& MT25220 \& MX35220 \& CSM0120}

\begin{center}
\begin{huge}
Lecture 10: Modules and Documentation
\end{huge}
\bigskip

Amanda Clare (afc@aber.ac.uk)

\end{center}
\end{frame}

\begin{frame} 
\frametitle{Spyder and help}
Demo of the variable explorer, the file explorer and the object explorer on Spyder.
\end{frame}

\begin{frame} 
\frametitle{Setting the working directory for the console}
Use the Global Working Directory dropdown (in the bar at the top) to
navigate to where you want to be. This setting will apply to any new
consoles (not to existing consoles).

Open a new console (from the menu bar, choose the Consoles menu).

\bigskip

But the easiest way to set the working directory is to just run your code (green button). 
\end{frame}

\begin{frame} 
\frametitle{Main}
\begin{itemize}
\item Now we can write functions and call them.
\item We aim to have all the code in functions, so it is all
reusable.
\item We still need some place that actually does
something, that calls the functions.
\item This is often (by convention) a function called \texttt{main}.
\end{itemize}
\end{frame}

\begin{frame}[fragile]
\frametitle{Previously: weather.py}
The zip file on Blackboard contains two files: weather.py and
predict.py. 

Download and unzip, and open weather.py (or use the one
you had previously if it's ok).
\end{frame}

\begin{frame}[fragile]
\frametitle{Previously: weather.py}
\begin{code} 
import random

def predict_tomorrow_temp(current_temp):
   change = random.choice(range(-1, 2))
   new_temp = current_temp + change
   return new_temp

temp = 10
for day in range(1, 8):
   print("Day " + str(day))
   print("Today's temp is:" + str(temp))
   temp = predict_tomorrow_temp(temp)
   print("Tomorrow will be:" + str(temp))
\end{code}
\end{frame}


\begin{frame}[fragile]
\frametitle{Add a main}
\begin{code} 
import random

def predict_tomorrow_temp(current_temp):
   change = random.choice(range(-1, 2))
   new_temp = current_temp + change
   return new_temp

def main():
   temp = 10
   for day in range(1, 8):
      print("Day " + str(day))
      print("Today's temp is:" + str(temp))
      temp = predict_tomorrow_temp(temp)
      print("Tomorrow will be:" + str(temp))

main()
\end{code}
\end{frame}

\begin{frame}[fragile]
\frametitle{Check whether we should call main}
\begin{code} 
import random

def predict_tomorrow_temp(current_temp):
   change = random.choice(range(-1, 2))
   new_temp = current_temp + change
   return new_temp

def main():
   temp = 10
   for day in range(1, 8):
      print("Day " + str(day))
      print("Today's temp is:" + str(temp))
      temp = predict_tomorrow_temp(temp)
      print("Tomorrow will be:" + str(temp))

if __name__ == "__main__":
   main()
\end{code}
\end{frame}

\begin{frame}[fragile]
\frametitle{Using your module}
Your code can now be:
\begin{enumerate}
\item imported as a module (providing two functions
\texttt{predict\_tomorrow\_temp} and \texttt{main}) 
\item or run as a script.
\end{enumerate}
\end{frame}


\begin{frame}[fragile]
\frametitle{predict.py}
Open the downloaded 'predict.py'.

\bigskip

\begin{code} 
import weather

today = 10
tomorrow = weather.predict_tomorrow_temp(today)
print("Tomorrow will be:" + str(tomorrow))
msg = weather.weather_message(tomorrow, 5, 15)
print(msg)
\end{code}
\end{frame}

\begin{frame}[fragile]
\frametitle{Give predict.py a main function too}
\begin{code}
import weather

def main():
   today = 10
   tomorrow = weather.predict_tomorrow_temp(today)
   print("Tomorrow will be:" + str(tomorrow))
   msg = weather.weather_message(tomorrow, 5, 15)
   print(msg)

if __name__ == "__main__":
   main()
\end{code}
\end{frame}

\begin{frame}[fragile]
\frametitle{import  ... as ...}

If you don't want to type the full module name:
 
\begin{code}
import weather as wtr

def main():
   today = 10
   tomorrow = wtr.predict_tomorrow_temp(today)
   print("Tomorrow will be:" + str(tomorrow))
   msg = wtr.weather_message(tomorrow, 5, 15)
   print(msg)

if __name__ == "__main__":
   main()
\end{code}
\end{frame}

\begin{frame}[fragile]
\frametitle{Comments}
\begin{itemize}
\item We should add lots of explanatory comments to code.
\item Comments are for humans. The computer will ignore them.
\item Comments start with a \# (``hash'', ``octothorpe'')
\end{itemize}

\bigskip

 Comments can be on a line by themselves
\begin{code}
# This loop will add all the bird
# data into the species table
\end{code}

\bigskip

Or they can be at the end of a line of code
\begin{code}
x = 16 # initial value known to be too high
\end{code}
\end{frame}

\begin{frame}[fragile]
\frametitle{Comments should be useful}
No hard and fast rule on what is useful.
\begin{code}
# This initialises x
x = 16

# We need a loop
for y in range(1,5):
   print(birds[y])
\end{code}

\bigskip

\begin{code}
# x represents the number of birds
# We initialise it to a value known to be the
# minimum number submitted by the yearly report.
x = 16
\end{code}
\end{frame}

\begin{frame}[fragile]
\frametitle{Docstrings}
Docstrings are triple-quoted strings:

\begin{code}
"""This is a docstring"""
"""This is another"""
\end{code}

They are used to provide documentation for your module and
functions.

We already use comments:

\begin{code}
# this is a comment and will be ignored by Python
# but is useful for human readers of the code
\end{code}

But docstrings offer something different.
\end{frame}

\begin{frame}[fragile]
\frametitle{docexample.py}
Have a look at the code in the file docexample.py.
Run the code.

\bigskip

Load the code as a module:
\begin{code}
>>> import docexample
>>> help(docexample)
>>> help(docexample.ten_times)
\end{code}
\end{frame}



\begin{frame}[fragile]
\frametitle{docexample.py}
The help that you see comes from the docstrings. We put
them:
\begin{itemize}
\item at the start of the file to describe the module
\item at the start of each function to describe the function
\end{itemize} 
Python then uses these to provide help for your module.

\end{frame}

\begin{frame}[fragile]
\frametitle{See the documentation shown by the help function}
\begin{code}
>>> import docexample
>>> help(docexample)
>>> help(docexample.ten_times)

>>> import math
>>> help(math.pow)

>>> import matplotlib.pyplot
>>> help(matplotlib.pyplot.bar)
\end{code}
\end{frame}

\begin{frame}[fragile]
\frametitle{pydoc can produce a web page}
It will take the docstrings in your code and produce a web page
describing your module.
\begin{code}
>>> import pydoc
>>> import docexample
>>> pydoc.writedoc(docexample)
\end{code}
produces a file called:
\texttt{docexample.html}. Open it in a web browser and have a look.
\end{frame}

\begin{frame}[fragile]
\frametitle{What kind of documentation is useful?}
\begin{itemize}
\item Short description.
\item Longer description.
\item Description of parameters
\item Return value
\item Any exceptions thrown
\item Notes
\item See also
\item Examples
\end{itemize}
\end{frame}

\begin{frame}[fragile]
\frametitle{Summary}
\begin{itemize}
\item Write all code in functions
\item Collect functions in a module/file
\item Can import the module 
\item If needed to run as a script, add a main function
\item Add lots of comments
\item Use triple-quoted docstrings to document the module
\item Can generate a web page of documentation with pydoc
\end{itemize}
\end{frame}


\end{document}

