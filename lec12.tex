\documentclass{beamer}

\usepackage{listings}
\usepackage{color}
\usepackage{hyperref}

\usepackage[T1]{fontenc}
\usepackage{textcomp}
\usepackage{upquote}

% Default fixed font does not support bold face
\DeclareFixedFont{\ttb}{T1}{txtt}{bx}{n}{10} % for bold
\DeclareFixedFont{\ttm}{T1}{txtt}{m}{n}{10}  % for normal

% Custom colors
\usepackage{color}
\definecolor{deepblue}{rgb}{0,0,0.5}
\definecolor{deepred}{rgb}{0.6,0,0}
\definecolor{deepgreen}{rgb}{0,0.5,0}
\definecolor{shadecolor}{rgb}{1, 0.9, 0.3}

\usepackage{listings}

% Python style for highlighting
\newcommand\pythonstyle{\lstset{
language=Python,
basicstyle=\ttm,
otherkeywords={self},             % Add keywords here
keywordstyle=\ttb\color{deepblue},
emph={MyClass,__init__},          % Custom highlighting
emphstyle=\ttb\color{deepred},    % Custom highlighting style
stringstyle=\color{deepgreen},
frame=tb,                         % Any extra options here
showstringspaces=false,            % 
upquote=True,
columns=fullflexible,
basicstyle=\ttfamily
}}


% Python environment
\lstnewenvironment{code}[1][]
{
%\begin{small}
\pythonstyle
\lstset{#1}
%\end{small}
}
{}


\begin{document}

% Hands on objects (some code with simple objects that have a few methods and vars to play with).

\begin{frame}
\frametitle{CS24420 \& MA25220 \& MT25220 \& MX35220 \& CSM0120}

\begin{center}
\begin{huge}
Lecture 12: Objects - the basics 
\end{huge}
\bigskip

Alexander Pitchford (agp1@loadaber.ac.uk)

\end{center}
\end{frame}


\begin{frame}[fragile]
\frametitle{Introduction}
Classes define objects\\
Objects have \emph{methods} (functions)\\
and \emph{attribites} (or \emph{properties})\\
\bigskip
Often we have many objects of the same class each with different attributes  
\end{frame}

\begin{frame}[fragile]
\frametitle{Preparation}
The zip file \texttt{lec12\_code.zip} on Blackboard contains a number of files for
use today. Some are python files we will edit, some are data files containing names.

Download, save in the folder where you keep your python work, and unzip.

Open Spyder - choosing the correct one as in previous weeks.

Open \texttt{washing.py}



\end{frame}

\begin{frame}[fragile]
\frametitle{Testing the WashingMachine 1}
This is a toy example, in the sense that it does not really do anything useful.
You can't wash your clothes with a computer program!

Run the file \texttt{washing.py} in a Python Console. There will be no output, 
but it will set your Working Directory. 

Now run these lines in the Python Console
(remember you can use up and down arrows to select previously run lines)
\begin{code} 
mach1 = WashingMachine()
mach1.load("dirty clothes")
print("machine contains: {}".format(mach1.contents))
mach1.wash()
print("machine contains: {}".format(mach1.contents))
\end{code}

\end{frame}

\begin{frame}[fragile]
\frametitle{Testing the WashingMachine 2}
You may be disappointed to see that your clothes are still dirty.

Try adding these lines
\begin{code} 
mach1.add_detergent()
mach1.wash()
print("machine contains: {}".format(mach1.contents))
\end{code}

\texttt{mach1} is an object of class \texttt{WashingMachine}.
\texttt{load}, \texttt{wash} and \texttt{add\_detergent}
are all methods (functions) defined by the class and hence available to the object.
\texttt{contents} is an attribute.

\end{frame}

\begin{frame}[fragile]
\frametitle{Defining a WasherDryer}
Now we will define a \texttt{WasherDryer} class by \emph{subclassing} the \texttt{WashingMachine}

Add these lines to the end of the \texttt{washing.py} file (not the Console!)
\begin{code} 
class WasherDryer(WashingMachine):
    """Machine for washing and drying clothes"""
    def reset(self):
        """Reset all attributes to default values"""
        WashingMachine.reset(self)
        self.model = 'generic washer dryer'
        self.dry_time = 1.0
\end{code}

\end{frame}

\begin{frame}[fragile]
\frametitle{Add a 'dry' method}
Now we will add a new method for our new class.

Add these lines to the \texttt{washing.py} file. Note the indentation - 
it should be at the same level as the \texttt{reset} method
\begin{code} 
    def dry(self):
        print("Drying...")
        sleep(self.dry_time)
        for ww in self.wet_words:
            if ww in self.contents:
                self.contents = self.contents.replace(ww, 
                                                   'dry')
        print("Dry cycle finished")
\end{code}
\end{frame}

\begin{frame}[fragile]
\frametitle{Testing the WasherDryer}

Now run these lines in the Python Console

\begin{code} 
wd1 = WasherDryer()
wd1.load("dirty clothes")
print("machine contains: {}".format(wd1.contents))
wd1.add_detergent()
wd1.wash()
print("machine contains: {}".format(wd1.contents))
wd1.dry()
print("machine contains: {}".format(wd1.contents))
\end{code}

\end{frame}

\begin{frame}[fragile]
\frametitle{A specific model}

There is another a class in the file that defines a specific model.
It's faster than the generic model. 
Try it out - run these lines in the Python Console

\begin{code} 
zan1 = ZanussiZWF91483W()
zan1.load("dirty clothes")
print("machine contains: {}".format(zan1.contents))
zan1.add_detergent()
zan1.wash()
print("machine contains: {}".format(zan1.contents))
zan1.dry()
print("machine contains: {}".format(zan1.contents))
\end{code}
The \texttt{NotImplementedError} is expected. Hopefully it is informative.

\end{frame}

\begin{frame}[fragile]
\frametitle{Optional Excercise - wash and dry}

If you are skipping ahead and want something extra to do,
then try adding another method \texttt{wash\_and\_dry} to
the \texttt{WasherDryer} class.

You can call the other methods (of this class and the \emph{superclass})
using:

\begin{code} 
        self.wash()
        self.dry()
\end{code}

\end{frame}

%-----------------------------------------------------------------------------------
\begin{frame}[fragile]
\frametitle{A useful example - people}

The next example is a more realistic use of classes and objects.
Open the file \texttt{people.py}. 

This contains definitions for classes \texttt{Person} and \texttt{PersonalStatistics}

This time we will create a Python script file to try out the classes. 
A file \texttt{people\_play.py} has been provided for you to add to. Open this one too.

As the classes and functions are in separate file you will need to import them. 
Start by adding this line to \texttt{people\_play.py}

\begin{code} 
from people import (Person, generate_random_people, 
                    PersonalStatistics)
\end{code}

\end{frame}

\begin{frame}[fragile]
\frametitle{Me and my friends}
The \texttt{Person} class requires that some attributes are set when an object is created.
This is done by passing arguments when you create a \texttt{Person} object.
These are the name, sex, age and height respectively.
Try creating a few people, starting with yourself and few friends. 
Then add them all to a list. (this is just an example, chose your own names and personal attributes)

\begin{code} 
alex = Person('Alex Pitchford', 'male', 42, 1.75)
fred = Person('Fred Blogs', 'male', 55, 1.9)
bilbo = Person('Bilbo Baggins', 'male', 131, 0.9)
meandmyfriends = [alex, fred, bilbo]
for p in meandmyfriends:
    print(p)
\end{code}
Run the script to test it.

\end{frame}


\begin{frame}[fragile]
\frametitle{Me and my friends - stats}
The \texttt{PersonalStatistics} class will accept a list people when it is created.
Try creating one for you and your friends

\begin{code} 
mmf_stats = PersonalStatistics(meandmyfriends)
mmf_stats.report()
\end{code}
Run the script to test it.
You should see the \texttt{report} method provides some statistics 
on the split of the sexes and on the ages.

\end{frame}

\begin{frame}[fragile]
\frametitle{More people}
Now lets try with some bigger data sets. We will use the \texttt{generate\_random\_people}
function that we imported earlier.

\begin{code} 
peeps10 = generate_random_people(10)
stats10 = PersonalStatistics(peeps10)
stats10.report()
peeps100 = generate_random_people(100)
stats100 = PersonalStatistics(peeps100)
stats100.report()
peeps10000 = generate_random_people(10000)
stats10000 = PersonalStatistics(peeps10000)
stats10000.report()
\end{code}
Run the script to test it.
You mostly likely see that for the bigger data sets the splits and means approach the distribution, 
which are 49\%, 49\%, 2\% and 60 years old.

\end{frame}

\begin{frame}[fragile]
\frametitle{Me and my friends - stats}
The \texttt{PersonalStatistics} class will accept a list people when it is created.
Try creating one for you and your friends

\begin{code} 
mmf_stats = PersonalStatistics(meandmyfriends)
mmf_stats.report()
\end{code}
Run the script to test it.
You should see the \texttt{report} method provides some statistics 
on the split of the sexes and on the ages.

\end{frame}

%----------------------------------------------
\begin{frame}[fragile]
\frametitle{more stats - standard deviation}
Now lets add method for calculating the standard deviation of the age
For this we need to edit the class - that's in the \texttt{people.py} file.
Add this method beneath the \texttt{mean\_age} method.
\begin{code} 
    def std_age(self):
        """Calculate the standard deviation
         of age of the people in the list"""
        return np.std([p.age for p in self.people])
\end{code}
And this line to the \texttt{report} method
\begin{code} 
        print("Std dev age: {:0.2f}".format(
                                self.std_age()))
\end{code}
Run the script \texttt{people\_play.py} to test the new addition.

\end{frame}

\begin{frame}[fragile]
\frametitle{more stats - height}
Now we will add stats for the height.
Add these methods to the \texttt{PersonalStatistics} class
\begin{code} 
    def mean_height(self):
        """Calculate the mean average height 
        of the people in the list"""
        return np.mean([p.height for p in self.people])
        
    def std_height(self):
        """Calculate the standard deviation of 
        height of the people in the list"""
        return np.std([p.height for p in self.people])
\end{code}

\end{frame}

\begin{frame}[fragile]
\frametitle{report height stats}
And these lines to the \texttt{report} method
\begin{code} 
        print("Mean average height: {:0.2f}".format(
                                  self.mean_height()))
        print("Std dev height: {:0.2f}".format(
                                   self.std_height()))
\end{code}
Run the script \texttt{people\_play.py} to test the new additions.

\end{frame}

\end{document}
