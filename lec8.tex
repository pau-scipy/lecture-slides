\documentclass{beamer}

\usepackage{listings}
\usepackage{color}
\usepackage{hyperref}

% Default fixed font does not support bold face
\DeclareFixedFont{\ttb}{T1}{txtt}{bx}{n}{10} % for bold
\DeclareFixedFont{\ttm}{T1}{txtt}{m}{n}{10}  % for normal

% Custom colors
\usepackage{color}
\definecolor{deepblue}{rgb}{0,0,0.5}
\definecolor{deepred}{rgb}{0.6,0,0}
\definecolor{deepgreen}{rgb}{0,0.5,0}
\definecolor{shadecolor}{rgb}{1, 0.8, 0.3}

\usepackage{listings}
\usepackage{framed}

% Python style for highlighting
\newcommand\pythonstyle{\lstset{
language=Python,
basicstyle=\ttm,
otherkeywords={self},             % Add keywords here
keywordstyle=\ttb\color{deepblue},
emph={MyClass,__init__},          % Custom highlighting
emphstyle=\ttb\color{deepred},    % Custom highlighting style
stringstyle=\color{deepgreen},
frame=tb,                         % Any extra options here
showstringspaces=false            % 
}}


% Python environment
\lstnewenvironment{code}[1][]
{
%\begin{small}
\pythonstyle
\lstset{#1}
%\end{small}
}
{}


\begin{document}

% Functions hands on. Remember existing functions eg len, sorted, sqrt and numpy functions. Define and call some functions with simple params and return values.

\begin{frame}
\frametitle{CS24420 \& MA25220 \& MT25220}

\begin{center}
\begin{huge}
Lecture 8: Hands-on Using functions 
\end{huge}
\bigskip

Amanda Clare (afc@aber.ac.uk)

\end{center}
\end{frame}


\begin{frame}[fragile]
\frametitle{Functions}
Functions allow us to group useful code statements
together for reusability.

\bigskip

You can use Python's built in functions, such
as len and np.random.random:
\begin{code}
>>> len( [1,2,5,6,7,3] )
>>> len("hello world")
>>> import numpy as np
>>> x = np.random.random((5, 10))
\end{code}
But you can also write your own.
\end{frame}


\begin{frame}[fragile]
\frametitle{weather\_message}
For example, here's a function called \texttt{weather\_message}
\begin{code}
import math
def weather_message():
   temp = 40 * math.sin(2.5)
   if temp < 3:
      msg = "wear wool trousers"
   elif temp < 10:
      msg = "wear trousers"
   else: 
      msg = "wear shorts"
   return msg
\end{code}
\pause
We write \texttt{def}, the function name, two brackets and a
colon.
The body of the function is indented.
\end{frame}

\begin{frame}[fragile]
\frametitle{Save the code in a file called weather.py}
\begin{shaded}
\begin{code}
import math
def weather_message():
   temp = 40 * math.sin(2.5)
   if temp < 3:
      msg = "wear wool trousers"
   elif temp < 10:
      msg = "wear trousers"
   else: 
      msg = "wear shorts"
   return msg
\end{code}
\end{shaded}
\begin{code}
>>> import weather
>>> w = weather.weather_message()
>>> print(w)
\end{code}
\end{frame}


\begin{frame}[fragile]
\frametitle{Your function can take a parameter}
\begin{shaded}
\begin{code}
def weather_message(temp):
  if temp < 3:
      msg = "wear wool trousers"
   elif temp < 10:
      msg = "wear trousers"
   else: 
      msg = "wear shorts"
   return msg
\end{code}
\end{shaded}
\begin{code}
>>> import weather
>>> w = weather.weather_message(15)
>>> print(w)
>>> w = weather.weather_message(8)
>>> print(w)
\end{code}
\end{frame}

\begin{frame}[fragile]
\frametitle{Many parameters}
\begin{shaded}
\begin{code}
def weather_message(temp, low, med):
  if temp < low:
      msg = "wear wool trousers"
   elif temp < med:
      msg = "wear trousers"
   else: 
      msg = "wear shorts"
   return msg
\end{code}
\end{shaded}
\begin{code}
>>> import weather
>>> w = weather.weather_message(15,5,20)
>>> print(w)
\end{code}
\end{frame}

\begin{frame}[fragile]
\frametitle{Reusable code}
We can then call that function many times and reuse it.
\begin{code}
>>> import weather
>>> w = weather.weather_message(6, 3, 10)
>>> w = weather.weather_message(100, 0, 45)
>>> w = weather.weather_message(4, 5, 6)
\end{code}
Writing code that is reusable is a fundamental aim for
programmers.

\bigskip

Others can use your code.

\bigskip

You can reuse your code in future projects.
\end{frame}

\begin{frame}[fragile]
\frametitle{Functions in a script}
\begin{shaded}
\begin{code}
def weather_message(temp, low, med):
   if temp < low:
      msg = "wear wool trousers"
   elif temp < med:
      msg = "wear trousers"
   else:
      msg = "wear shorts"
   return msg

w = weather_message(15, 6, 20)
print(w)
\end{code}
\end{shaded}
We can write functions and then call them, all in the
same file. We call this a ``script''. Note how the last two lines are
not indented.

\bigskip

We can execute this code at the command line:\\
\texttt{python weather.py}
\end{frame}


\begin{frame}[fragile]
\frametitle{Exercise - calling a function}

Write a for-loop to go through temperatures from 0 to
100 and call your \texttt{weather\_message} function for
each temperature. Always call it with a \texttt{low} of 7 and a
\texttt{med} of 15.
\bigskip

Just start writing your for-loop at the bottom of your
weather.py file.
\bigskip

Run the code with\\
\texttt{python weather.py}
\end{frame}


\begin{frame}[fragile]
\frametitle{Adding randomness}
Let's add another function to your weather.py file
\begin{shaded}
\begin{code}
import random
def predict_tomorrow_temp():
   return random.choice(range(-3, 25))
\end{code}
\end{shaded}

This function returns a value using the keyword return.\\
This function takes no parameters.\\
The value returned will be a random number in the range -3
up to 25 (but not including 25).\\
\begin{code}
>>> weather.predict_tomorrow_temp()
>>> t = weather.predict_tomorrow_temp()
>>> t
\end{code}
\end{frame}

\begin{frame}[fragile]
\frametitle{Using that result}
\begin{code}
>>> weather.predict_tomorrow_temp()
2
>>> weather.predict_tomorrow_temp()
9
>>> t = weather.predict_tomorrow_temp()
>>> print(t)
23
>>> t = weather.predict_tomorrow_temp()
>>> print(t)
-2
\end{code}
\end{frame}

\begin{frame}[fragile]
\frametitle{Taking a parameter}
\begin{shaded}
\begin{code}
import random
def predict_tomorrow_temp(today_temp):
   change = random.choice(range(-1, 2))
   new_temp = today_temp + change
   return new_temp
\end{code}
\end{shaded}
This function takes one parameter (\texttt{today\_temp})
and returns a result (\texttt{new\_temp}).
\end{frame}

\begin{frame}[fragile]
\frametitle{Using a parameter}
\begin{shaded}
\begin{code}
import random
def predict_tomorrow_temp(today_temp):
   change = random.choice(range(-1, 2))
   new_temp = today_temp + change
   return new_temp

today = 10
tomorrow = predict_tomorrow_temp(today)
next_day = predict_tomorrow_temp(tomorrow)
print(today)
print(tomorrow)
print(next_day)
\end{code}
\end{shaded}
\end{frame}

\begin{frame}[fragile]
\frametitle{Predict a whole week}
\begin{shaded}
\begin{code}
import random
def predict_tomorrow_temp(today_temp):
   change = random.choice(range(-1, 2))
   new_temp = today_temp + change
   return new_temp

temp = 10
for day in range(1,8):
   print("Day " + str(day))
   print("Today's temp is:" + str(temp))
   temp = predict_tomorrow_temp(temp)
   print("Tomorrow will be:" + str(temp))
\end{code}
\end{shaded}
\end{frame}

\begin{frame}[fragile]
\frametitle{Exercise}
\begin{shaded}
\begin{code}
import random
def predict_tomorrows_temp(today_temp):
   change = random.choice(range(-1,2))
   new_temp = today_temp + change
   return new_temp
\end{code}
\end{shaded}

How would you fix this function so that it never returned
temperatures lower than -3 or higher than 24? (Hint, you'll
need to use an if-statement).
\end{frame}

\begin{frame}
\frametitle{Summary}
\begin{itemize}
\item Writing functions
\item Calling functions
\item Functions can take parameters
\item Functions can return values that can be saved into variables
\item Functions allow you to write reusable code
\end{itemize}
\end{frame}

\end{document}
