\documentclass{beamer}

\usepackage{listings}
\usepackage{color}
\usepackage{hyperref}

% Default fixed font does not support bold face
\DeclareFixedFont{\ttb}{T1}{txtt}{bx}{n}{10} % for bold
\DeclareFixedFont{\ttm}{T1}{txtt}{m}{n}{10}  % for normal

% Custom colors
\usepackage{color}
\definecolor{deepblue}{rgb}{0,0,0.5}
\definecolor{deepred}{rgb}{0.6,0,0}
\definecolor{deepgreen}{rgb}{0,0.5,0}

\usepackage{listings}

% Python style for highlighting
\newcommand\pythonstyle{\lstset{
language=Python,
basicstyle=\ttm,
otherkeywords={self},             % Add keywords here
keywordstyle=\ttb\color{deepblue},
emph={MyClass,__init__},          % Custom highlighting
emphstyle=\ttb\color{deepred},    % Custom highlighting style
stringstyle=\color{deepgreen},
frame=tb,                         % Any extra options here
showstringspaces=false            % 
}}


% Python environment
\lstnewenvironment{code}[1][]
{
%\begin{small}
\pythonstyle
\lstset{#1}
%\end{small}
}
{}


\begin{document}

% Python basics hands on lecture. Types and variables. if-then.

\begin{frame}
\frametitle{CS24420 \& MA25220 \& MT25220}

\begin{center}
\begin{huge}
Lecture 2: Hands-on Types, Variables, if-else
\end{huge}
\bigskip

Amanda Clare (afc@aber.ac.uk)

\end{center}
\end{frame}


\begin{frame}[fragile]
\frametitle{Try it in your hello.py file}
Open Spyder, load the hello.py file you previously made and change the
code to the following:
\begin{code}
x = 1 + 5
y = 6/4.0
print(x)
print(y)
print(x + y)
\end{code}
\end{frame}

\begin{frame}[fragile]
\frametitle{Try it}
\begin{code}
x = "Hello"
y = "world"
print(x)
print(y)
print(x + y)
\end{code}
\end{frame}


\begin{frame}[fragile]
\frametitle{Try it}
\begin{code}
x = 4
y = "world"
print(x)
print(y)
print(x + y)
\end{code}
\end{frame}

\begin{frame}[fragile]
\frametitle{Try it}
\begin{code}
x = "Mary"
print(Mary)
\end{code}
\end{frame}



\begin{frame}
\frametitle{Basic types}
\begin{itemize}
\item {\tt NoneType}
\item {\tt bool}
\item {\tt int} (also {\tt long}, {\tt bignum})
\item {\tt float}
\item {\tt str} 
\end{itemize}
(also {\tt complex}, {\tt unicode}, and other less basic types)
\end{frame}

\begin{frame}[fragile]
\frametitle{Try it}
\begin{code}
>>> type(5)
>>> type(None)
>>> type(5.5)
>>> type(False)
>>> type("hello")
>>> type('hello')
>>> type('5.5')
\end{code}
\end{frame}

\begin{frame}[fragile]
\frametitle{Convert types}
\begin{code}
>>> int("5")
>>> bool(2)
>>> str(7.5)
>>> int(4.6)
>>> int("Wednesday")
>>> float(4)
>>> bool("hello")
\end{code}
\end{frame}


\begin{frame}[fragile]
\frametitle{Try it}
\begin{code}
>>> "a" + "b"
>>> 1 + 2
>>> "a" + 2
>>> "a" + str(2)
>>> 1 + "2"
>>> "6" / "4"
>>> "a" * 5
\end{code}
\end{frame}

\begin{frame}
\frametitle{Types}
\begin{itemize}
\item The {\tt NoneType} type has only one value, {\tt None}
\item The {\tt bool} type has two values, {\tt True} and {\tt False}
\item int, float, str have many possible values
\end{itemize}
\end{frame}

\begin{frame}[fragile]
\frametitle{Strings}
Single or double quotes: {\tt "hello" 'hello'}

Triple quotes:
\begin{code}
"""this string
spans many
lines
"""
\end{code}

Strings can behave as if they were a list of characters, so
you can use {\tt len} and {\tt in}: 
\begin{code}
>>> x = "hello world"
>>> len(x)
>>> "o" in x
>>> "wo" in x
\end{code}
\end{frame}


\begin{frame}[fragile]
\frametitle{Strings}

More things we can do with strings:
\begin{code}
>>> s = "ATGCTTATACA"
>>> s.count('A')
>>> s.startswith("AT")
>>> s.endswith("GC")
>>> s.find("G")
>>> s.find("N")
>>> s = s + "NNNNN"
\end{code}
\end{frame}

\begin{frame}[fragile]
\frametitle{if - else}
\begin{code}
temp = 40 * math.sin(2.5)
if temp < 10:
   msg = "wear trousers"
else:
   msg = "wear shorts"
print(msg)
\end{code}
We need to be able to make decisions and change the
outcome based on the decisions.
\begin{itemize}
\item Note the indentation and the colons ( {\tt :} )
\item {\tt if} needs a condition to test ({\tt temp < 10})
\item The condition should evaluate to {\tt True} or {\tt False}
\item {\tt else} doesn't have a condition
\end{itemize}
\end{frame}

\begin{frame}[fragile]
\frametitle{if - else}
\begin{code}
temp = 40 * math.sin(2.5)
if temp < 10:
   msg = "Feeling cold! Only " + str(temp) + "."
   msg = msg + "Wear trousers"
else:
   msg = "We're enjoying summer."
   msg = msg + "Wear shorts"
print(msg)
\end{code}
We can have many lines of code within each indented
block.
\end{frame}


\begin{frame}[fragile]
\frametitle{if - elif - else}
\begin{code}
temp = 40 * math.sin(2.5)
if temp < 3:
   msg = "wear wool trousers"
elif temp < 10:
   msg = "wear trousers"
else:
   msg = "wear shorts"
  
print(msg)
\end{code}
\begin{itemize}
\item if and elif need a condition to test
\item else doesn't have a condition
\end{itemize}
\end{frame}

\begin{frame}[fragile]
\frametitle{if - else}
\begin{code}
temp = 40 * math.sin(2.5)
rain = 10
if temp < 10:
   msg = "Feeling cold! Only " + str(temp)
   if rain > 5:
      msg = msg + "Wear waterproof trousers"
   else:
      msg = msg + "Wear trousers"
else:
   msg = "We're enjoying summer"
   msg = msg + "wear shorts"
\end{code}
\begin{itemize}
\item We can have {\tt if} statements within other {\tt if}
statements.
\item They {\bf need} good indentation
\end{itemize}
\end{frame}

\begin{frame}[fragile]
\frametitle{Conditions}
What kinds of condition can we have?

Generally want something that evaluates to {\tt True} or {\tt False}.

\begin{code}
x < y
x <= y
x >= y
x == y    # equality test uses a double equals sign!
x != y
x == 5
x < 3.5/2
x < "banana"
"apple" < "banana"
\end{code}
\end{frame}



\begin{frame}[fragile]
\frametitle{Conditions}
We can use {\tt and} and {\tt or} and {\tt not}.

\begin{code}
x = "Aberystwyth"
y = 3.0/5
if len(x) < 4 and y > 7:
   print("success")

x = len("Aberystwyth") > 5
y = True
if x or y:
   print("hooray!")

x = len("Aberystwyth") > 5
if not x:
   print("not long enough")
\end{code}
\end{frame}

\begin{frame}
\frametitle{Summary}
\begin{itemize}
\item Types
\item Saving values in variables
\item if-elif-else
\end{itemize}
\end{frame}

\end{document}
