\documentclass{beamer}

\usepackage{listings}
\usepackage{color}
\usepackage{hyperref}

% Default fixed font does not support bold face
\DeclareFixedFont{\ttb}{T1}{txtt}{bx}{n}{10} % for bold
\DeclareFixedFont{\ttm}{T1}{txtt}{m}{n}{10}  % for normal

% Custom colors
\usepackage{color}
\definecolor{deepblue}{rgb}{0,0,0.5}
\definecolor{deepred}{rgb}{0.6,0,0}
\definecolor{deepgreen}{rgb}{0,0.5,0}

\usepackage{listings}

% Python style for highlighting
\newcommand\pythonstyle{\lstset{
language=Python,
basicstyle=\ttm,
otherkeywords={self},             % Add keywords here
keywordstyle=\ttb\color{deepblue},
emph={MyClass,__init__},          % Custom highlighting
emphstyle=\ttb\color{deepred},    % Custom highlighting style
stringstyle=\color{deepgreen},
frame=tb,                         % Any extra options here
showstringspaces=false            % 
}}


% Python environment
\lstnewenvironment{code}[1][]
{
%\begin{small}
\pythonstyle
\lstset{#1}
%\end{small}
}
{}


\begin{document}

% Practical lecture: Lists, hands on examples

\begin{frame}
\frametitle{CS24420 \& MA25220 \& MT25220}

\begin{center}
\begin{huge}
Lecture 4: Hands-on Lists 
\end{huge}
\bigskip

Amanda Clare (afc@aber.ac.uk)

\end{center}
\end{frame}


\begin{frame}[fragile]
\frametitle{Lists}
\begin{code}
>>> x = [1,3,4,5,7,2]
>>> y = ["hello","hi","shwmae","morning"]
>>> z = [True, False, True]
>>> w = []
>>> n = len(x)
\end{code}
\end{frame}




\begin{frame}[fragile]
\frametitle{Lists}
\begin{code}
>>> x = [1,3,4,5,7,2]
\end{code}
 List elements are counted from position 0
\begin{code}
>>> x[0]
>>> x[2]
\end{code}
The last element in x is at what position?
\end{frame}

\begin{frame}[fragile]
\frametitle{Lists}
\begin{code}
>>> x = [1,3,4,5,7,2]
\end{code}
Which element will this change?
\begin{code}
>>> x[2] = 8
>>> print(x)
\end{code}
What did you see?
\end{frame}

\begin{frame}[fragile]
\frametitle{Lists}
What will this do?
\begin{code}
>>> x = [1,3,4,5,7,2]
>>> x[0] = x[5]
\end{code}
\bigskip
What will this do?
\begin{code}
>>> x[10] = 4
>>> x.append(8)
\end{code}
\end{frame}

\begin{frame}[fragile]
\frametitle{Joining lists together}
We can add two lists together to make a longer list.
\begin{code}
>>> x = [1, 2, 3]
>>> y = [4, 5, 6]
>>> z = x + y
>>> print(z)
\end{code}
 I can extend y by x. What happens to y and to x?
\begin{code}
>>> y.extend(x)
>>> print(y)
>>> print(x)
\end{code}
 If I now change part of y, will x be altered?
\begin{code}
>>> y[4] = 9
>>> print(y)
>>> print(x)
\end{code}
\end{frame}

\begin{frame}[fragile]
\frametitle{Slicing lists}
Slicing is the name for just copying some sub part of a list.
For example, if we have a list called x then we can use:
\begin{code}
x[1:4] # items at positions 1, 2 and 3 (but not 4)
x[2:6] # items at positions 2, 3, 4 and 5
x[2:] # all items from position 2 to the end 
x[:4] # the first 4 items (in positions 0, 1, 2 and 3)
\end{code}

\begin{code}
>>> x = [2,3,4,5,6,7,8]
>>> y = x[1:4]
>>> y[0] = 0
>>> print(y)
>>> print(x)
\end{code}
\end{frame}

\begin{frame}[fragile]
\frametitle{Splitting strings}
\begin{code}
>>> x = "2,3,5,y,n,large"
>>> y = x.split(",")
>>> print(y)

>>> x = "hello and welcome to python"
>>> y = x.split()
>>> print(y)
\end{code}

\texttt{split()} with no arguments will split on whitespace
\end{frame}


\begin{frame}[fragile]
\frametitle{Joining strings}
Joining a list of strings into a single string:
\begin{code}
>>> x = "hello and welcome to python"
>>> y = x.split()
>>> z = ",".join(splitX)
>>> print(z)
\end{code}

Joining can be done with any string separator:
\begin{code}
>>> x = ["1","2","3","4"]
>>> spacedX = " ".join(x)
>>> tabbedX = "\t".join(x)
>>> commaX = ",".join(x)
>>> wooX = "woo".join(x)
\end{code}
\end{frame}

\begin{frame}[fragile]
\frametitle{Looping over a list}
\begin{code}
for x in ['aberystwyth', 'bangor', 'cardiff','swansea']:
   print(x + str(len(x)))
\end{code}

The variable \texttt{x} is used to represent each element of the list
in turn.

What does this code do?

Why is str present?
\end{frame}

\begin{frame}[fragile]
\frametitle{Exercises}
1. Read in a sentence of text from the keyboard and report the number of
words that were typed.

\bigskip

2. Read in a sentence of text from the keyboard and display the longest word
that was typed.

\bigskip 

Hints: you'll need to split the sentence into a list of words. Remember
how to do user input from lecture 3 with sys.stdin.readline().
\end{frame}


\begin{frame}[fragile]
\frametitle{Are strings and lists the same?}
\begin{code}
>>> x = len([4,3,2,5,6])
>>> x = len("hello")

>>> sorted([4,3,2,5,6])
>>> sorted("hello")

>>> x = [4,3,2,5,6]
>>> x[3] = 9

>>> x = "hello"
>>> x[3] = 'f'
\end{code}

\end{frame}


\begin{frame}[fragile]
\frametitle{Modifying lists, or not?}
\begin{code}
>>> x = [3,5,6,8,2,9]
>>> y = sorted(x)
>>> x.sort()
\end{code}

sorted() will return a new list. sort() will return None.

\bigskip

Some functions in Python will destructively modify lists (for
efficiency), and some will not. 
\end{frame}

\begin{frame}[fragile]
\frametitle{Modifying lists, or not?}
\begin{code}
>>> x = [3,5,6,8,2,9]
>>> x.reverse()
>>> x[::-1]
>>> reversed(x)
\end{code}

x.reverse() will destructively modify the list in place and return None. x[::-1] will
return a new list. reversed(x) will return a generator. 

\bigskip

Some functions in Python will destructively modify lists (for
efficiency), and some will not. 
\end{frame}


\end{document}
