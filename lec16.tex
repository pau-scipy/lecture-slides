\documentclass{beamer}

\usepackage{listings}
\usepackage{color}
\usepackage{hyperref}

\usepackage[T1]{fontenc}
\usepackage{textcomp}
\usepackage{upquote}


% Default fixed font does not support bold face
\DeclareFixedFont{\ttb}{T1}{txtt}{bx}{n}{10} % for bold
\DeclareFixedFont{\ttm}{T1}{txtt}{m}{n}{10}  % for normal

% Custom colors
\usepackage{color}
\definecolor{deepblue}{rgb}{0,0,0.5}
\definecolor{deepred}{rgb}{0.6,0,0}
\definecolor{deepgreen}{rgb}{0,0.5,0}
\definecolor{shadecolor}{rgb}{1, 0.9, 0.3}

\usepackage{listings}

% Python style for highlighting
\newcommand\pythonstyle{\lstset{
language=Python,
basicstyle=\ttm,
otherkeywords={self, as},             % Add keywords here
keywordstyle=\ttb\color{deepblue},
emph={MyClass,__init__},          % Custom highlighting
emphstyle=\ttb\color{deepred},    % Custom highlighting style
stringstyle=\color{deepgreen},
frame=tb,                         % Any extra options here
showstringspaces=false,            % 
upquote=True,
columns=fullflexible,
basicstyle=\ttfamily
}}


% Python environment
\lstnewenvironment{code}[1][]
{
%\begin{small}
\pythonstyle
\lstset{#1}
%\end{small}
}
{}


\begin{document}

% Hands on with plotting graphs using data from files.

\begin{frame}
\frametitle{CS24420 \& MA25220 \& MT25220 \& MX35220 \& CSM0120}

\begin{center}
\begin{huge}
Lecture 16: Plotting data
\end{huge}
\bigskip

Amanda Clare (afc@aber.ac.uk)

\end{center}
\end{frame}


\begin{frame}[fragile]
\frametitle{matplotlib}
\begin{itemize}
\item matplotlib provides MATLAB-like 2D plotting.
\item A free and open source library for Python. \url{http://matplotlib.org}
\item Produces a range of image formats: png, pdf, ps, eps, svg etc
\begin{figure}
        \centering
        \begin{minipage}{.5\textwidth}
            \centering
            \includegraphics[width=5cm]{histogram_demo_multihist.png}
       \end{minipage}%
        \begin{minipage}{.5\textwidth}
            \centering
            \includegraphics[width=5cm]{finance_work2.png}
       \end{minipage}
   \end{figure}
\end{itemize}

\end{frame}

\begin{frame}[fragile]
\frametitle{The most basic plot}
\begin{code}
import matplotlib.pyplot as plt
plt.plot([1, 3, 2, 4])
plt.ylabel("the y axis name")
plt.show()
\end{code}
\end{frame}

\begin{frame}[fragile]
\frametitle{Adding x values for the y values}
\begin{code}
import matplotlib.pyplot as plt
plt.plot([10, 20, 30, 40], [1, 3, 2, 4])
plt.ylabel("the y axis name")
plt.xlabel("some x axis")
plt.show()
\end{code}
\end{frame}

\begin{frame}[fragile]
\frametitle{Plotting triangles etc}
\begin{code}
import matplotlib.pyplot as plt

plt.plot([10, 20, 30, 40], [1, 3, 2, 4], "bv")
plt.ylabel("the y axis name")
plt.xlabel("some x axis")
plt.show()
\end{code}

\bigskip

Try \texttt{"ro" "g." "c-" "k:"}
\end{frame}

\begin{frame}[fragile]
\frametitle{Plotting points and lines}
Create variables x and y to hold the data. Plot with points and plot
again with lines on the same graph.
\begin{code}
import matplotlib.pyplot as plt

x = [10, 20, 30, 40]
y = [1, 3, 2, 4]

plt.plot(x, y, "bv")
plt.plot(x, y, "b-")
plt.ylabel("the y axis name")
plt.xlabel("some x axis")
plt.show()
\end{code}
\end{frame}

\begin{frame}[fragile]
\frametitle{Plotting multiple data series on one graph}
\begin{code}
import matplotlib.pyplot as plt

x = [10, 20, 30, 40]
y1 = [1, 3, 2, 4]
y2 = [1.5, 2.5, 3.5, 4.5]

plt.plot(x, y1, "b-") 
plt.plot(x, y2, "r-")
plt.axis([0, 50, 0, 5])
plt.ylabel("the y axis name")
plt.xlabel("some x axis")
plt.show()
\end{code}
\end{frame}

\begin{frame}[fragile]
\frametitle{Saving the image}
\begin{code}
import matplotlib.pyplot as plt

x = [1, 2, 3, 4]
mouse = [10, 12, 20, 15]

plt.bar(x, mouse)
plt.xticks(x, ('A', 'T', 'C', 'G'))
plt.savefig("fig.png", dpi=200)
\end{code}
\pythonstyle{\lstinline{plt.savefig("fig.png")}}

or

\pythonstyle{\lstinline{plt.savefig("fig.pdf")}}

or

\pythonstyle{\lstinline{plt.savefig("fig.tiff")}}
\end{frame}

\begin{frame}[fragile]
\frametitle{Saving the image}
\includegraphics[width=10cm]{barfig.png}
\end{frame}

\begin{frame}[fragile]
\frametitle{Midpoints of the bars}
We'd prefer the bar ticks to be at the midpoints of the bars. So we
need to calculate where the midpoints of the bars are.

If the x locations are [1, 2, 3, 4] and the bars are of width 0.8 then
the midpoints of the bars are [1.4, 2.4, 3.4, 4.4].

\begin{code}
import matplotlib.pyplot as plt

x = [1, 2, 3, 4]
mouse = [10, 12, 20, 15]

barwidth = 0.8
midpoints = []
for b in x:
   midpoints.append(b + barwidth/2)
plt.bar(x, mouse, barwidth)
plt.xticks(midpoints, ('A', 'T', 'C', 'G'))
plt.show()
\end{code}
\end{frame}

\begin{frame}[fragile]
\frametitle{Multiple bars}
\begin{code}
import matplotlib.pyplot as plt
x = [1, 2, 3, 4]
mouse = [10, 12, 20, 15]
human = [11, 14, 5, 5]
barwidth = 0.4
midpoints = []
for b in x:
   midpoints.append(b + barwidth)
plt.bar(x, mouse, barwidth, color="r", label="mouse")
plt.bar(midpoints,human,barwidth,color="b",label="human")
plt.xticks(midpoints, ('A', 'T', 'C', 'G'))
plt.legend()
plt.grid(True, axis='y')
plt.show()
\end{code}
\end{frame}

\begin{frame}[fragile]
\frametitle{Multiple bars}
\includegraphics[width=10cm]{twobars.png}
\end{frame}


\begin{frame}[fragile]
\frametitle{Scatter plot exercise}
Read the documentation for scatter plots:

\url{http://matplotlib.org/api/pyplot_api.html?highlight=scatter#matplotlib.pyplot.scatter}

and produce a scatter plot with green diamond
markers for the following data:
\begin{code}
import random
x = range(0,100)
y = random.sample(range(1000),100)
\end{code}
\end{frame}

\begin{frame}[fragile]
\frametitle{Summary}
Begun to experiment with:
\begin{itemize}
\item Examples of plots: plot, bar, scatter
\item Customising: labels, axes, colors, markers, positions
\item Many more matplotlib plot styles on offer 
\end{itemize}
\end{frame}

\end{document}
