\documentclass{beamer}

\usepackage{listings}
\usepackage{color}
\usepackage{hyperref}

% Default fixed font does not support bold face
\DeclareFixedFont{\ttb}{T1}{txtt}{bx}{n}{10} % for bold
\DeclareFixedFont{\ttm}{T1}{txtt}{m}{n}{10}  % for normal

% Custom colors
\usepackage{color}
\definecolor{deepblue}{rgb}{0,0,0.5}
\definecolor{deepred}{rgb}{0.6,0,0}
\definecolor{deepgreen}{rgb}{0,0.5,0}

\usepackage{listings}

% Python style for highlighting
\newcommand\pythonstyle{\lstset{
language=Python,
basicstyle=\ttm,
otherkeywords={self},             % Add keywords here
keywordstyle=\ttb\color{deepblue},
emph={MyClass,__init__},          % Custom highlighting
emphstyle=\ttb\color{deepred},    % Custom highlighting style
stringstyle=\color{deepgreen},
frame=tb,                         % Any extra options here
showstringspaces=false            % 
}}


% Python environment
\lstnewenvironment{code}[1][]
{
%\begin{small}
\pythonstyle
\lstset{#1}
%\end{small}
}
{}


\begin{document}

% Lists and dictionaries theory. For loops and lists. Quiz on content so far. 

\begin{frame}
\frametitle{CS24420 \& MA25220 \& MT25220}

\begin{center}
\begin{huge}
Lecture 5: Lists and Dictionaries
\end{huge}
\bigskip

Amanda Clare (afc@aber.ac.uk)

\end{center}
\end{frame}


\begin{frame}[fragile]
\frametitle{Dictionaries}
A dictionary is a data structure where you can store
keys and values.

For example:
\begin{itemize}
\item A birthday book: look up the birthday (value) for the
name of a person (key)
\item A price list: look up the price (value) of a product
using its barcode (key)
\item Coordinates of a gene: look up the location (value)
of a gene given its systematic identifier (key)
\item Look up the capital city (value) given the name of a
country
\end{itemize}
\end{frame}

\begin{frame}[fragile]
\frametitle{Creating dictionaries and adding items}
We use curly brackets to create a new dictionary:
\begin{code}
>>> my_dict = {}
\end{code}
We can add items (new key and value):
\begin{code}
>>> my_dict['John'] = 724736
\end{code}
We can update existing items (alter the value):
\begin{code}
>>> my_dict['John'] = 666666
\end{code}
We could create a new dictionary with many items:
\begin{code}
>>> my_dict = { "a": 34, "b":28, "c":56 }
\end{code}
\end{frame}

\begin{frame}[fragile]
\frametitle{Requesting data}
We can request the value that the dictionary holds for a specific
key:
\begin{code}
>>> my_dict = { "John" : 4, "Sarah" : 5 }
>>> my_dict["John"]
\end{code}
 What happens if a key is not present?
\begin{code}
>>> my_dict["Jane"]
\end{code}
 What happens if a key is used more than once?
\begin{code}
>>> my_dict = { "John" : 4, "Sarah" : 5, "John" : 7 }
>>> my_dict["John"]
\end{code}
\end{frame}


\begin{frame}[fragile]
\frametitle{Requesting data}
We can check if a key is present, and if so, find out the
value:
\begin{code}
student = "Jane"
if student in my_dict:
   print(my_dict[student])
else:
   print("No value found for " + student)
\end{code}
\end{frame}

\begin{frame}[fragile]
\frametitle{Requesting data}
We can check if a key is present, and if so, find out the
value:
\begin{code}
student = "Jane"
if student in my_dict:
   print(my_dict[student])
else:
   print("No value found for " + student)
\end{code}
\end{frame}

\begin{frame}
\frametitle{Exercise}
Create an empty dictionary.

Add the following information to that dictionary: the
name (keys) and favourite food (values) of the three
people sitting closest to you. If their names are not
unique, use their full names.

Check that each person's favourite food can be looked
up in the dictionary.
\end{frame}

\begin{frame}[fragile]
\frametitle{Getting values out}
\begin{code}
my_dict = { "a": 34, "b":28, "c":56 }
print(my_dict.keys())
for k in my_dict.keys():
   print(my_dict[k])
\end{code}
 We use .keys() to find out the keys that are in a dictionary.
We use square brackets after the dictionary name [] to find
out what value belongs to that key.

\bigskip

If we're not sure if a key is present, better check:
\begin{code}
if my_key in my_dict:
   print(my_dict[my_key])
else:
   print(my_key + " is not there")
\end{code}
\end{frame}

\begin{frame}[fragile]
\frametitle{Quiz (in teams)}
bar
\begin{code}
x = 1 + 5
\end{code}
\end{frame}


\end{document}
