\documentclass{beamer}

\usepackage{listings}
\usepackage{color}
\usepackage{hyperref}

\usepackage[T1]{fontenc}
\usepackage{textcomp}
\usepackage{helvet}
\usepackage{upquote}

% Default fixed font does not support bold face
\DeclareFixedFont{\ttb}{T1}{txtt}{bx}{n}{10} % for bold
\DeclareFixedFont{\ttm}{T1}{txtt}{m}{n}{10}  % for normal

% Custom colors
\usepackage{color}
\definecolor{deepblue}{rgb}{0,0,0.5}
\definecolor{deepred}{rgb}{0.6,0,0}
\definecolor{deepgreen}{rgb}{0,0.5,0}
\definecolor{purple}{RGB}{153, 0, 153}

\usepackage{listings}

% Python style for highlighting
\newcommand\pythonstyle{\lstset{
language=Python,
basicstyle=\ttm,
otherkeywords={self},             % Add keywords here
keywordstyle=\ttb\color{deepblue},
emph={MyClass,__init__},          % Custom highlighting
emphstyle=\ttb\color{deepred},    % Custom highlighting style
stringstyle=\color{deepgreen},
frame=tb,                         % Any extra options here
showstringspaces=false,            % 
}}


% Python environment
\lstnewenvironment{code}[1][]
{
%\begin{small}
\pythonstyle
\lstset{#1}
%\end{small}
}
{}


\begin{document}

% Lists and dictionaries theory. For loops and lists. Quiz on content so far. 



\begin{frame}
\frametitle{CS24420 \& MA25220 \& MT25220 \& MX35220 \& CSM0120}

\begin{center}
\begin{huge}
Lecture 5\\
Tuples, Dictionaries and Sets
\end{huge}
\bigskip

Amanda Clare (afc@aber.ac.uk)\\
Sam Nicholls (msn@aber.ac.uk)

\end{center}
\end{frame}

\begin{frame}[fragile]
\frametitle{Tuples}
    A \texttt{tuple} is a \textbf{finite} sequence of \textbf{ordered} items.\\
    \texttt{Tuple}s store a \textbf{fixed} number of \textbf{immutable} elements.\\
    \texttt{Tuple}s can serve as a container for grouping related variables.
    \\For example:

    \begin{itemize}
        \item A co-ordinate such as (x, y), or (latitude, longitude)
        \item A colour defined in RGB space, like {\color{purple}(153, 0, 153)}
        \item Grouped information such as (firstname, surname, age)
    \end{itemize}

\vskip 0.3cm
    \texttt{Tuple}s look like \texttt{list}s, but are enclosed in parentheses:
\begin{code}
>>> aber = (52.4153, -4.0829)
>>> sam = ("Sam", "Nicholls", 25)
\end{code}
Although valid, it's considered bad practice to mix types in a \texttt{list}.
\texttt{Tuple}s provide a more natural data structure for mixed types.
\end{frame}


\begin{frame}[fragile]
\frametitle{Handling \texttt{tuples}}
    Like \texttt{list}s, \texttt{tuple} elements can be indexed, sliced and counted:
\begin{code}
>>> sam = ("Sam", "Nicholls", 25)
>>> sam[0]
'Sam'
>>> sam[1:3]
('Nicholls', 25)
>>> len(sam)
3
\end{code}

\vskip 0.3cm
Unlike \texttt{list}s, elements of \texttt{tuple}s are \textbf{immutable}:
\begin{code}
>>> sam[2] = 26
TypeError: 'tuple' object does not support
           item assignment
\end{code}
\end{frame}


\begin{frame}[fragile]
\frametitle{Tuples as Containers}
    \texttt{Tuple}s may contain other data structures, including \texttt{tuple}s!

\begin{code}
>>> sam = ("Sam", "Nicholls", ["cs244"])
>>> alice = ("Alice", "Wunderland", ["en207"])
>>> students = (sam, alice)
>>> students
(
    ('Sam', 'Nicholls', ['cs244']),
    ('Alice', 'Wunderland', ['en207'])
)
\end{code}
%TODO should I use students[0][2] instead?
Although one cannot assign to individual elements of a \texttt{tuple}, \textbf{mutable}
elements such as \texttt{list}s can be modified:
\begin{code}
>>> alice[2].append("ma252")
>>> sam[2][0] = "csm01"
>>> students
(('Sam', 'Nicholls', ['cms01']),
('Alice', 'Wunderland', ['en207', 'ma252']))
\end{code}
\end{frame}


\begin{frame}[fragile]
\frametitle{Packing and Unpacking Tuples}
\texttt{Tuple}s can actually be constructed without the parentheses...\\
A \texttt{tuple} can be \textit{packed} by enumerating its elements
and delimiting them with a comma:

\begin{code}
>>> sam = "Sam", "Nicholls", 25
>>> sam
('Sam', 'Nicholls', 25)
\end{code}

\vskip 0.3cm
Conversely, one may \textit{unpack} each of the elements of a \texttt{tuple}
    into suitably named, \textbf{ordered} variables:
\begin{code}
>>> firstname, surname, age = sam
>>> firstname
'Sam'
\end{code}
Unpacking requires the same number of variables on the left as there
    are elements in the \texttt{tuple} on the right.
\end{frame}

\begin{frame}[fragile]
\frametitle{Dictionaries}
    A \texttt{dict}ionary is a data structure where you can store
    mappings between \textbf{keys} and \textbf{values}.

For example:
\begin{itemize}
\item A birthday book: look up the birthday (value) for the
name of a person (key)
\item A price list: look up the price (value) of a product
using its barcode (key)
\item Coordinates of a gene: look up the location (value)
of a gene given its systematic identifier (key)
\item Look up the capital city (value) given the name of a
country
\end{itemize}
\end{frame}

\begin{frame}[fragile]
\frametitle{Creating dictionaries and adding items}
We use curly brackets to create a new dictionary:
\begin{code}
>>> phonebook = {}
\end{code}

We can add items (new key and value):
\begin{code}
>>> phonebook['John'] = 724736
\end{code}

We can update existing items (alter the value):
\begin{code}
>>> phonebook['John'] = 666666
\end{code}

We can delete existing items:
\begin{code}
>>> del phonebook['John']
\end{code}

We could create a new dictionary with many items:
\begin{code}
>>> ages = { "a": 34, "b":28, "c":56 }
\end{code}
\end{frame}

\begin{frame}[fragile]
\frametitle{Requesting data}
We can request the value that the dictionary holds for a specific
key:
\begin{code}
>>> office = { "John" : 4, "Sarah" : 5 }
>>> office["John"]
\end{code}
 What happens if a key is not present?
\begin{code}
>>> office["Jane"]
\end{code}
 What happens if a key is used more than once?
\begin{code}
>>> office = { "John" : 4, "Sarah" : 5, "John" : 7 }
>>> office["John"]
\end{code}
\end{frame}


\begin{frame}[fragile]
\frametitle{Requesting data}
We can check if a key is present, and if so, find out the
value:
\begin{code}
student = "Jane"
if student in exam_result:
   print(exam_result[student])
else:
   print("No value found for " + student)
\end{code}
\end{frame}

\begin{frame}[fragile]
\frametitle{Getting all values out}
\begin{code}
age = { "a": 34, "b":28, "c":56 }
print(age.keys())
print(age.values())

for k in age.keys():
   print(age[k])
\end{code}
We can use .keys() to find out the keys that are in a dictionary.
We use square brackets after the dictionary name [] to find
out what value belongs to that key.
\end{frame}

\begin{frame}[fragile]
\frametitle{Simpler, more Pythonic}
\begin{code}
for k in my_dict.keys():
   print(my_dict[k])
\end{code}
can be written instead
\begin{code}
for k in my_dict:
   print(my_dict[k])
\end{code}
The .keys() is assumed if we're looping over a dictionary.
\end{frame}


\begin{frame}[fragile]
\frametitle{Ordering of \texttt{dict}ionary keys}
\texttt{Dict}ionaries are \textbf{unordered} data structures.\\
The ordering of keys is non-random, but arbitrary.
\begin{code}
>>> phonebook = {}
>>> phonebook["msn"] = 4443
>>> phonebook["afc"] = 2429
>>> phonebook["waa2"] = 2421 
>>> phonebook.keys()
['msn', 'waa2', 'afc']
\end{code}
The keys \textbf{are not ordered} alphanumerically, neither are they in the same
order in which they were inserted into the \texttt{dict}ionary.
\vskip 0.3cm

Use \texttt{sorted()} to iterate over \texttt{dict}ionaries in a sorted fashion:
\begin{code}
for k in sorted(phonebook):
   print(phonebook[k])
\end{code}
\end{frame}


\begin{frame}[fragile]
\frametitle{Iterating \texttt{dict}ionaries with \texttt{tuple}s}
We can get a \texttt{list} of (key, value) \texttt{tuples} of the elements
in a \texttt{dict}ionary with \texttt{items()}:
\begin{code}
>>> phonebook.items()
[('msn', 4443), ('waa2', 2421), ('afc', 2429)]
\end{code}

\vskip 0.3cm
Instead of looking up the value for a key inside our loop using the dictionary,
we can \textit{unpack} the (key, value) tuples in the \texttt{for} loop:
\begin{code}
for key, value in sorted(phonebook.items()):
    print("User: " + key + " Phone: " + str(value))
\end{code}
\end{frame}

%todo hmm
\begin{frame}[fragile]
\frametitle{Sets}
    A \texttt{set} is a collection of \textbf{unordered}, \textbf{unique} elements.\\
    For example:

    \begin{itemize}
        \item All even numbers
        \item A hand of playing cards (dealt from a single deck)
        \item Students registered on a module
        \item Colours in a rainbow
        \item Results of a lottery draw
        \item Each term of the sequence $1/(2^i)$ for $\texttt{i}=0...\texttt{n}$
    \end{itemize}

    \texttt{Set}s are typically used as a data structure for fast 
    membership testing (\textit{is something already in this set?}), and for
    efficient elimination of duplicate elements from another collection.
\end{frame}


\begin{frame}[fragile]
\frametitle{Creating and updating \texttt{set}s}
\texttt{Set}s look like \texttt{list}s, but are enclosed in braces:
\begin{code}
>>> colours = {"red", "green", "blue"}
>>> lotto = {14, 16, 25, 42, 46, 51, 53}
\end{code}

\vskip 0.2cm
A \texttt{set} \textbf{cannot be indexed} (or sliced):
\begin{code}
>>> lotto[0]
TypeError: 'set' object does not support indexing
\end{code}

\vskip 0.2cm
    We can add additional values to a \texttt{set} with \texttt{add} and \texttt{update}:
\begin{code}
>>> colours.add("purple")
>>> colours.update(["yellow", "pink"])
>>> colours
set(['blue', 'purple', 'green', 'yellow', 'red', 'pink'])
\end{code}
\end{frame}


%todo woops used uppercase for set variable names...
\begin{frame}[fragile]
\frametitle{Membership and cardinality of \texttt{sets}}
An element of a \texttt{set} is said to be a \textbf{member} of that \texttt{set}.\\
    We say that $x$ is a member of \texttt{set} $A$, or that
    "x belongs to A".\\
    Mathematically we denote this: $x \in A$ (read $\in$ as "in")\\
\begin{code}
>>> A = {1, 2, 3}
>>> 2 in A
True
\end{code}

\bigskip
    The \textbf{cardinality} of a \texttt{set} is defined as the number of members that a \texttt{set} has.
    For some set $A$, we denote its cardinality $|A|$.\\
\begin{code}
>>> len(A)
3
\end{code}
\end{frame}

\begin{frame}[fragile]
\frametitle{Equality of \texttt{set}s}
Although a \texttt{set} can contain elements that have a natural ordering, such as
integers, the collection itself is \textbf{unordered}. Thus the equivalence of sets
is based on whether they contain the same elements, regardless of ordering.
\begin{code}
>>> {1, 2, 3} == {3, 2, 1}
True
>>> {1, 2, 3} == {3, 2, 1, 0}
False
\end{code}
\end{frame}


\begin{frame}[fragile]
\frametitle{Set Operations}
Consider:
\begin{code}
>>> A = {1, 2, 3}
>>> B = {3, 4, 5}
\end{code}

\vskip 0.4cm
\textbf{Intersection}: Members that appear in both A and B
\begin{code}
>>> A & B
set([3])
\end{code}

\vskip 0.4cm
\textbf{Union}: The unique collection of all members across A and B
\begin{code}
>>> A | B
set([1, 2, 3, 4, 5])
\end{code}
\end{frame}


\begin{frame}[fragile]
\frametitle{Set Operations}
Continue to consider:
\begin{code}
>>> A = {1, 2, 3}
>>> B = {3, 4, 5}
\end{code}

\vskip 0.4cm
\textbf{Difference}: Members in A that are not in B
\begin{code}
>>> A - B
set([1, 2])
\end{code}

\vskip 0.4cm
\textbf{Symmetric Difference}: Members of A and B not in intersection
\begin{code}
>>> A ^ B
set([1, 2, 4, 5])
\end{code}
\end{frame}


\begin{frame}[fragile]
\frametitle{Set Operations}
Introducing C for your consideration:
\begin{code}
>>> A = {1, 2, 3}
>>> B = {3, 4, 5}
>>> C = {1, 3, 6}
\end{code}

\vskip 0.4cm
We can chain these operations to manipulate more than two sets in an expression:
\begin{code}
>>> (A | B) | C
set([1, 2, 3, 4, 5, 6])
>>> (A & B) & C
set([3])
>>> (A & B) - C
set([])
>>> (A ^ B) ^ C
set([2, 3, 4, 5, 6])
\end{code}
\end{frame}

\begin{frame}[fragile]
%gens and iters?
\frametitle{Comprehensions}
An efficient method to create \texttt{list}s, \texttt{dict}ionaries and \texttt{set}s.

\vskip 0.2cm
Here is a \texttt{for} loop to append the first 20 square numbers to a \texttt{list}:

\begin{code}
square_numbers = []
for i in range(20):
    square_numbers.append(i**2)
\end{code}

\vskip 0.3cm
This is perfectly reasonable, but \textbf{comprehensions} allow us to collapse
this loop into one line of Python:
\begin{code}
[i**2 for i in range(20)]
\end{code}
\end{frame}


\begin{frame}[fragile]
\frametitle{Comprehensions: Why?}

Comprehensions are powerful. We can use expressions and \texttt{if}s.\\
Consider this code that creates a list of only even squares:
\begin{code}
even_squares = []
for i in range(20):
    current_square = i**2
    if current_square % 2 == 0:
        even_squares.append(current_square)
\end{code}

\vskip 0.3cm
We can just add a conditional to our comprehension:
\begin{code}
[i**2 for i in range(20) if i**2 % 2 == 0]
\end{code}
\end{frame}


\begin{frame}[fragile]
\frametitle{Examples of Comprehensions}
Create a \textbf{list} of the reciprocals of $1/(2^i)$ for $i=0...53$:
\begin{code}
>>> [1.0/(2**i) for i in range(54)]
\end{code}

\vskip 0.3cm
Given a \texttt{list} of item (barcode, quantity) \texttt{tuples},
create a \textbf{dictionary} to map each product
barcode to the total amount spent on each type of item:
\begin{code}
>>> basket = [(100, 10), (101, 2), (111, 1)]
>>> prices = {100: 1.0, 101: 2.75, 111: 8.5}
>>> {b[0]: b[1]*prices[b[0]] for b in basket}
\end{code}

\vskip 0.3cm
Given a \texttt{list} of student (name, module\_list) \texttt{tuples},
create a \textbf{set} containing a register of student names enrolled on "cs244":
\begin{code}
>>> students = [("Sam", ["cs244"]), ("Ali", ["ma252"])]
>>> {s[0] for s in students if 'cs244' in s[1]}
\end{code}
\end{frame}


\begin{frame}[fragile]
\frametitle{Comprehension Inception}
Here is some code with an \textit{outer loop} that iterates (with a \texttt{for})
over a \texttt{list} of student (name, module\_list) \texttt{tuples} and an
\textit{inner loop} that iterates over the \texttt{list} of modules of each student
to create a \textbf{set} of unique 20 credit modules studied by any student:

\begin{code}
students = [
    ("Sam", ["cs23820", "ma25220", "csd0100"]),
    ("Alice", ["en20710", "ma25220"])
]
cred20_module_set = set()
for student in students:
    for module in student[1]:
        if module.endswith("20"):
            cred_20_module_set.add(module)
\end{code}

\begin{code}
{mod for stu in students
     for mod in stu[1] if mod.endswith("20")}
\end{code}
\end{frame}


\begin{frame}[fragile]
\frametitle{String Formatting}
Previously we've taught to convert to strings and concatenate:
\begin{code}
>>> print("Hello, my favourite number is " + str(8))
\end{code}

\vskip 0.3cm
However, sometimes we want more control over string formatting.\\
A mini-language called "\%-formatting" exists to allow you to define how individual
values are presented when printed.
It looks like this:

\begin{code}
>>> name = "Sam"
>>> phone = 4443
>>> print("Name: %s - Phone: %d" % (name, phone))
Name: Sam - Phone: 4443
\end{code}

%todo quite clunky wording?
For strings, the modulus operator \%
is overloaded to pass a tuple of variables that are substituted (in order)
to each of the
special \% specifiers of the string, which define how to format that variable.
\end{frame}


%todo technically %s will take *any* type
\begin{frame}[fragile]
\frametitle{String Formatting}
For now, you might want to know these four format specifiers:
\begin{code}
>>> print("%d is for integers" % (8))
>>> print("%f is for floats" % (3.14))
>>> print("%s is for strings" % ("hoot"))
>>> print("%% prints a percentage sign" % ())
\end{code}

\vskip 0.3cm
Particularly, you should know how to print a \texttt{float} to a certain
    number of decimal places. We can prefix the \texttt{f} with \texttt{.N}
    to specify the precision of the float to \texttt{N} decimal places:
%>>> item_price = 1.2
%>>> item_quantity = 4
%>>> total = item_price * item_quantity
%>>> total
%4.8
\begin{code}
>>> total = 4.8
>>> print("Total Price: $%.2f" % (total))
Total Price: $4.80
\end{code}

{\tiny{These 'percentage formatters' are considered "old style" but are not deprecated. For information on the\\\vskip -0.15cm newer format() options see: \url{https://docs.python.org/3.4/library/string.html#format-string-syntax}}}
\end{frame}



%\begin{frame}
%\frametitle{Exercise on the board}
%Create an empty dictionary.

%\bigskip

%Add the following information to that dictionary: the
%name (keys) and favourite food (values) of the three
%people sitting closest to you. If their names are not
%unique, use their full names.

%\bigskip

%One of those people has recently discovered how wonderful olives
%are. Update the favourite food for this person.

%\bigskip

%Find out whether Hannah is in the dictionary or not. If she is, report
%her favourite food. If she isn't, report that she's not present.
%\end{frame}



%\begin{frame}[fragile]
%\frametitle{Quiz (in teams)}
%\end{frame}

\begin{frame}[fragile]
\frametitle{Summary}
\textbf{Tuples}: Ordered grouping of related variables
\begin{code}
>>> aber = (52.4153, 4.0829)
>>> sam = ("Sam", "Nicholls", 25)
\end{code}

\vskip 0.2cm
\textbf{Dictionaries}: A key-value mapping
\begin{code}
>>> phonebook = {"msn": 4443, "afc": 2429}
\end{code}

\vskip 0.2cm
\textbf{Sets}: Unordered and unique collection
\begin{code}
>>> colours = {"red", "green", "blue"}
>>> lotto = {14, 16, 25, 42, 46, 51, 53}
\end{code}

\begin{itemize}
    \item List, dict and set comprehensions
    \item String formatting
\end{itemize}
\end{frame}

\end{document}
