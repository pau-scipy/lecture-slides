\documentclass{beamer}

\usepackage{listings}
\usepackage{color}
\usepackage{hyperref}

\usepackage[T1]{fontenc}
\usepackage{textcomp}
\usepackage{upquote}


% Default fixed font does not support bold face
\DeclareFixedFont{\ttb}{T1}{txtt}{bx}{n}{10} % for bold
\DeclareFixedFont{\ttm}{T1}{txtt}{m}{n}{10}  % for normal

% Custom colors
\usepackage{color}
\definecolor{deepblue}{rgb}{0,0,0.5}
\definecolor{deepred}{rgb}{0.6,0,0}
\definecolor{deepgreen}{rgb}{0,0.5,0}
\definecolor{shadecolor}{rgb}{1, 0.9, 0.3}

\usepackage{listings}

% Python style for highlighting
\newcommand\pythonstyle{\lstset{
language=Python,
basicstyle=\ttm,
otherkeywords={self},             % Add keywords here
keywordstyle=\ttb\color{deepblue},
emph={MyClass,__init__},          % Custom highlighting
emphstyle=\ttb\color{deepred},    % Custom highlighting style
stringstyle=\color{deepgreen},
frame=tb,                         % Any extra options here
showstringspaces=false,            % 
upquote=True,
columns=fullflexible,
basicstyle=\ttfamily
}}


% Python environment
\lstnewenvironment{code}[1][]
{
%\begin{small}
\pythonstyle
\lstset{#1}
%\end{small}
}
{}


\begin{document}



\begin{frame}
\frametitle{CS24420 \& MA25220 \& MT25220 \& MX35220 \& CSM0120}

\begin{center}
\begin{huge}
Lecture 17: More Plotting
\end{huge}
\bigskip

Amanda Clare (afc@aber.ac.uk)

\end{center}
\end{frame}

\begin{frame}[fragile]
\frametitle{Until the end of term}
\begin{itemize}
\item \textbf{Tues 6th:} Plotting, and prac with worksheet 8. Sign off worksheet 6 deadline.
\item \textbf{Fri 9th:} Mathematical algorithms hack.
\item \textbf{Tues 13th:} Revision class. Prac, and sign off worksheet 7 deadline.
\item \textbf{Fri 16th:} Optional extra practical. Sign off worksheet 8 deadline.
\end{itemize}
January: Extra material on Databases and Python (using Sqlite).
\end{frame}


\begin{frame}[fragile]
\frametitle{Writing data to files}
\begin{code}
with open("foo.txt", "w") as output_file:
    output_file.write("whatever we want to write\n")
    for i in range(100):
        output_file.write(i)
\end{code}
\end{frame}


\begin{frame}[fragile]
\frametitle{Error bars}
\includegraphics[width=10cm]{errorbars.png}
\end{frame}


\begin{frame}[fragile]
\frametitle{Error bars}

We can plot error bars for each y value:

\begin{code}
x = [1, 2, 3, 4, 5]
y = [10, 12, 15, 22, 28]
errors = [2.2, 1.5, 1.8, 2.8, 1.4]

plt.errorbar(x, y, yerr=errors)

plt.axis([0, 6, 0, 40])
plt.ylabel('the y axis name')
plt.xlabel('some x axis')
plt.show()
\end{code}

Could also plot xerr (error in the x position). 
\end{frame}


\begin{frame}[fragile]
\frametitle{Error bars by themselves}
\includegraphics[width=10cm]{justerrors.png}
\end{frame}


\begin{frame}[fragile]
\frametitle{Error bars by themselves}

\begin{code}
x = [1, 2, 3, 4, 5]
y = [10, 12, 15, 22, 28]
errors = [2.2, 1.5, 1.8, 2.8, 1.4]

plt.errorbar(x, y, yerr=errors, fmt="none")

plt.axis([0, 6, 0, 40])
plt.ylabel('the y axis name')
plt.xlabel('some x axis')
plt.show()
\end{code}

In Python 2 this would have been \texttt{fmt=None}.
\end{frame}


\begin{frame}[fragile]
\frametitle{Error bars in colour}
\includegraphics[width=10cm]{errorscolour.png}
\end{frame}

\begin{frame}[fragile]
\frametitle{Error bars in colo\textcolor{red}{u}r}

Error bars in a different colo\textcolor{red}{u}r.

\begin{code}
x = [1, 2, 3, 4, 5]
y = [10, 12, 15, 22, 28]
errors = [2.2, 1.5, 1.8, 2.8, 1.4]
plt.plot(x, y, 'b-')
plt.plot(x, y, 'bo')

plt.errorbar(x, y, yerr=errors, fmt="none", ecolor="r")

plt.axis([0, 6, 0, 40])
plt.ylabel('the y axis name')
plt.xlabel('some x axis')
plt.show()
\end{code}
\end{frame}


% datetime plots
\begin{frame}[fragile]
\frametitle{Plotting datetime values}
Lets say we have a list of datetime objects. We can plot this against
a list of y values:
\begin{code}
plt.plot(dts, y, 'r-')

# Will tilt the date labels on the x axis
plt.gcf().autofmt_xdate()
\end{code}

\bigskip

\bigskip

\includegraphics[width=7cm]{datetime.png}

\end{frame}


\begin{frame}[fragile]
\frametitle{Plotting tides}

Downloaded tide heights for the past month for Castletownbere on the south west coast of Ireland from \url{http://data.marine.ie/Dataset/Details/20932} as a CSV file:

\bigskip

\begin{footnotesize}
\begin{verbatim}
station_id,time,Water_Level,Water_Level_LAT,Water_Level_OD_Malin,QC_Flag,longitude,latitude,altitude
,UTC,m,m,m,,degrees_east,degrees_north,m
Castletownbere,2016-11-06T00:00:00Z,1.968,1.968,-0.52,0.0,-9.9034,51.6496,0.0
Castletownbere,2016-11-06T00:15:00Z,1.888,1.888,-0.6,0.0,-9.9034,51.6496,0.0
Castletownbere,2016-11-06T00:30:00Z,1.808,1.808,-0.68,0.0,-9.9034,51.6496,0.0
Castletownbere,2016-11-06T00:45:00Z,1.758,1.758,-0.73,0.0,-9.9034,51.6496,0.0
\end{verbatim}
\end{footnotesize}

\end{frame}

\begin{frame}[fragile]
\frametitle{Plotting tides}
Code here for your notes but also provided as tides.py.

\begin{code}
import matplotlib.pyplot as plt
import csv
import datetime as dt
\end{code}

\end{frame}

\begin{frame}[fragile]
\frametitle{Plotting tides}

\begin{code}
def get_data_from_file(filename, dates, heights):
    format = "%Y-%m-%dT%H:%M:00Z"
    with open(filename,'r') as f:
        csvreader = csv.reader(f)
        next(csvreader) # throw away header line
        next(csvreader) # throw away second header line
        for row in csvreader:
            thedatetime = row[1]
            height = float(row[2])
            d_o = dt.datetime.strptime(thedatetime, format)
            dates.append(d_o)
            heights.append(height)
\end{code}
\end{frame}


\begin{frame}[fragile]
\frametitle{Plotting tides}

\begin{code}
def plot_tides(dates, heights):
    plt.plot(dates, heights, 'c-')
    plt.ylabel("Height in m")
    plt.gcf().autofmt_xdate()
    plt.savefig("tides.png")

if __name__ == "__main__":
    dates = []
    heights = [] 
    get_data_from_file(
       "IrishNationalTideGaugeNetwork_6b1b_95a2_06e9.csv", 
       dates, heights)
    plot_tides(dates, heights)
\end{code}
\end{frame}

\begin{frame}[fragile]
\frametitle{Plotting tides}
\includegraphics[width=12cm]{tides.png}
\end{frame}

%plotting loglog
\begin{frame}[fragile]
\frametitle{Plotting a relationship}

Linear relationship
\[ y = mx + c \]

Exponential relationship
\[ y = Ax^n \]


For example:
\[y = 5x +6\]

\[y = 5x ^3 \] 

We can generate data from these equations and plot them.
\end{frame}


\begin{frame}[fragile]
\frametitle{Functions to generate data}
\begin{code}
def gen_linear(m, c):
    """Generates data for y = mx + c"""
    x = np.linspace(0,10)
    y = m * x + c
    return (x, y)

def gen_exponential(a, n):
    """Generates data for y = ax^n"""
    x = np.linspace(0,10)
    y = a * x ** n
    return (x, y)
\end{code}

\end{frame}


\begin{frame}[fragile]
\frametitle{Plot it!}
\begin{code}
if __name__ == "__main__":

    (x, y) = gen_linear(5, 3)
    plt.plot(x, y, 'b-')
    plt.plot(x, y, 'bv')

    (x, y) = gen_exponential(5, 3)
    plt.plot(x, y, 'r-')
    plt.plot(x, y, 'rv')

    plt.savefig("linexp.png")
    plt.show()
\end{code}
\end{frame}

\begin{frame}[fragile]
\frametitle{Plot it!}
\includegraphics[width=12cm]{linexp.png}
\end{frame}

\begin{frame}[fragile]
\frametitle{Exponential relationship}
If we have an exponential relationship between variables x and y then when we take logs of both sides we will get a linear relationship. 
\[ y = Ax^n \]

\[log(y) = log(Ax^n) \]

\[ log(y) = log(A) + log(x^n) \]

\[ log(y) = log(A) + n log(x) \]

So plotting this data on log-transformed axes should show a straight line.

\end{frame}


\begin{frame}[fragile]
\frametitle{Plot exponential relationships on a loglog scale}
\begin{code}
def gen_exponential(a, n):
    """Generates data for y = ax^n"""
    x = np.linspace(0,10)
    y = a * x ** n
    return (x, y)

if __name__ == "__main__":
    (x, y) = gen_exponential(5, 3)
    plt.loglog(x, y, 'r-')
    plt.loglog(x, y, 'rv')
    plt.savefig("loglog.png")
    plt.show()
\end{code}
\end{frame}



\begin{frame}[fragile]
\frametitle{Plots with logarithmic axes}
\includegraphics[width=10cm]{loglog.png}
\end{frame}

\begin{frame}[fragile]
\frametitle{Richter scale}
The Richter scale for measuring earthquakes is a logarithmic scale.

\bigskip

Using data from the Wikipedia description of the comparison between the Richter scale and the equivalent mass of TNT that would be needed for some events:
\url{https://en.wikipedia.org/wiki/Richter_magnitude_scale}

\bigskip

\[ log( \textrm{amount of TNT}) \propto \textrm{Richter magnitude}\]
\end{frame}

\begin{frame}[fragile]
\frametitle{Plot with semi-logarithmic axes}
\includegraphics[width=11cm]{richter.png}

\end{frame}

\begin{frame}[fragile]
\frametitle{xkcd}
\includegraphics[width=8cm]{log_scale_xkcd.png}

\bigskip

Also see \url{https://xkcd.com/482/}: Universe on a log scale.
\end{frame}

\begin{frame}[fragile]
\frametitle{When using graphs in a report:}
\begin{itemize}
\item Label both axes of every graph (description and units).
\item Caption the graph in full (not just 3 words). Do this even if you
feel you are repeating what you said in the body of the report. ``Figure 1. A graph of tide heights over the course of a month beginning 8th November 2016 for Castletownbere, Ireland. Data was obtained from http://data.marine.ie/. Note that the variation in height of the high and low tides over the month is approximately 1 m.'' 
\item If presenting a graph on screen, explain what the graph
shows.
``This is a graph of population size in London through the
Victorian era in Britain.
The x-axis shows the years from 1840 to 1900.
The y-axis shows population in hundreds.
I'm showing this graph to point out how population in major
cities rose sharply during the industrial revolution and
especially so after 1860 because of ...''
\end{itemize}
\end{frame}


\begin{frame}[fragile]
\frametitle{Summary}
More plotting:
\begin{itemize}
\item Error bars
\item Dates
\item Log scale graphs
\end{itemize}
\end{frame}


\end{document}
