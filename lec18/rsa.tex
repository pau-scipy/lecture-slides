\documentclass[a4paper]{article}

% Margins to 2cm
\usepackage[left=2cm,right=2cm,top=3cm,bottom=2cm]{geometry}

% No identation on paragraphs
\usepackage{parskip}

\usepackage[T1]{fontenc}
\usepackage{textcomp}
\usepackage{upquote}
\usepackage{graphicx}
\usepackage{amsmath}

% Allow for code samples and change captions for "Listings" to "Code Snippet"
% \usepackage{minted}
\usepackage{xcolor}
\usepackage{listings} %bera
\definecolor{keywords}{RGB}{0,0,255}
\definecolor{comments}{RGB}{190,190,190}
\lstset{basicstyle=\ttfamily,
language=Python,
  showstringspaces=false,
  commentstyle=\color{red},
  keywordstyle=\color{blue},
  upquote=True
}

% URLS
\usepackage{hyperref}

% Prevent hypenation (not the nicest way to do this)
\hyphenpenalty 1000

% Asterism
\newcommand{\asterism}{\bigskip\par\centerline{*\,*\,*}\medskip\par}%

% Page headers
\usepackage{fancyhdr}

\pagestyle{fancy}
\fancyhf{}
\rhead{Mathematical Algorithms Hack}
\lhead{CS244{\tiny{20}} - CSM01{\tiny{20}} - MA252{\tiny{20}} - MT252{\tiny{20}} - MX352{\tiny{20}}}
\rfoot{\thepage}

% Use a sensible encoding so we don't have fancy quotes
\usepackage[T1]{fontenc}

%%%%%%%%%%%%%%%%%%%%%%%%%%%%%%%%%%%%%%%%%%%%%%%%%%%%%%%
%For hiding and showing solutions:

\usepackage{etoolbox}   % for booleans and much more
\usepackage{verbatim}   % for the comment environment

% setup a new boolean
\newbool{hidesolution}
\setbool{hidesolution}{false}

% new environment
\newenvironment{solution}{}{}

% set conditional behaviour of environment
\ifbool{hidesolution}{\AtBeginEnvironment{solution}{\comment}%
\AtEndEnvironment{solution}{\endcomment}}{}
%%%%%%%%%%%%%%%%%%%%%%%%%%%%%%%%%%%%%%%%%%%%%%%%%%%%%%%


\begin{document}
\begin{center}
    {\huge{Rivest-Shamir-Adleman (RSA) encryption}}\\

\end{center}

\vskip 0.5cm

RSA is a public key encryption system. If Alice wants to send an
encrypted message to Bob, she uses Bob's public key to encrypt the
message. Bob can then use his private key to decrypt the message. Even
though the public key can be made public and shared among everyone, it
is still very difficult to decrypt the message without the private
key. The difficulty is based on factoring a product of two (usually
large) prime numbers.

We want to generate suitable public and private keys, encrypt a message and
then decrypt the message.

You have been given a template for some code: \texttt{rsa.py}. There
are several functions that need finishing. These currently just return
None, but should instead return something useful.

\section*{Generate keys}
\begin{itemize}
\item First we ask the user for two prime numbers, $p$ and $q$. We compute $n$, the product
of $p$ and $q$. We now need to find $t$, which is Euler's totient of $n$. Euler's
totient of $n$ is the number of positive integers less than or equal to $n$
that are relatively prime to $n$. See
\url{http://mathworld.wolfram.com/TotientFunction.html}. You should
fill in the details of the function \texttt{euler\_totient}. 

\item Next we need to choose a suitable public key $e$. 
  We will choose an integer $e$ such that $1
  < e < t$ and e and t are coprime. We can do this by choosing a
  $e$ to be a prime number which is less than $t$ and does not divide
  $t$. Fill in the function \texttt{make\_primes} to generate all
  prime numbers between 2 and $t$. Fill in the function
  \texttt{choose\_e} to choose one of these primes and ensure it does
  not divide $t$.

\item Now we need to calculate the private key $d$. This will be the
  modular inverse of $e$ (modulo $t$). A function
  \texttt{modular\_inverse} is already provided. Fill in the function \texttt{calc\_d}.

\end{itemize}

\section*{Encrypting and decrypting}
Finally we can test that encryption and decryption work. If we
  have chosen $n$, $e$ and $d$ successfully then we can see that for any
  message $m$:

\[ c = m^e \mod n \] will encrypt the message, producing $c$.

\[ m = c^d \mod m \] will decrypt the encrypted message, producing
$m$. This is because $d$ was chosen to be the modular inverse of $e$.

\[ (m^d)^e = m \mod n\]

This code is already provided.

Can you encrypt a message, with another team's public key ($e$,$n$)
and then have them decrypt it using their private key?

How could you work out $p$ and $q$ from a public key ($e$, $n$) and hence crack someone else's encryption?
\end{document}