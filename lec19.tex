\documentclass{beamer}

\usepackage{listings}
\usepackage{color}
\usepackage{hyperref}

\usepackage[T1]{fontenc}
\usepackage{textcomp}
\usepackage{upquote}


% Default fixed font does not support bold face
\DeclareFixedFont{\ttb}{T1}{txtt}{bx}{n}{10} % for bold
\DeclareFixedFont{\ttm}{T1}{txtt}{m}{n}{10}  % for normal

% Custom colors
\usepackage{color}
\definecolor{deepblue}{rgb}{0,0,0.5}
\definecolor{deepred}{rgb}{0.6,0,0}
\definecolor{deepgreen}{rgb}{0,0.5,0}
\definecolor{shadecolor}{rgb}{1, 0.9, 0.3}

\usepackage{listings}

% Python style for highlighting
\newcommand\pythonstyle{\lstset{
language=Python,
basicstyle=\ttm,
otherkeywords={self},             % Add keywords here
keywordstyle=\ttb\color{deepblue},
emph={MyClass,__init__},          % Custom highlighting
emphstyle=\ttb\color{deepred},    % Custom highlighting style
stringstyle=\color{deepgreen},
frame=tb,                         % Any extra options here
showstringspaces=false,            % 
upquote=True,
columns=fullflexible,
basicstyle=\ttfamily
}}


% Python environment
\lstnewenvironment{code}[1][]
{
%\begin{small}
\pythonstyle
\lstset{#1}
%\end{small}
}
{}


\begin{document}



\begin{frame}
\frametitle{CS24420 \& MA25220 \& MT25220 \& MX35220 \& CSM0120}

\begin{center}
\begin{huge}
Lecture 19: Revision
\end{huge}
\bigskip

Amanda Clare (afc@aber.ac.uk)

\end{center}
\end{frame}

\begin{frame}[fragile]
\frametitle{Until the end of term}
\begin{itemize}
\item \textbf{Tues 13th:} Revision class. Prac, and sign off worksheet 7 deadline.
\item \textbf{Fri 16th:} Optional extra practical. Sign off worksheet 8 deadline.
\end{itemize}
January: Extra material on Databases and Python (using Sqlite).
\end{frame}

\begin{frame}[fragile]
\frametitle{L-systems}
\begin{itemize}
\item  Invented by Lindenmayer (Hungarian theoretical biologist).

\item Formal grammar for the transformation of strings using rules, and then
the translation of the strings into shapes.

\item For modelling plant growth, or decsribing fractals.
\end{itemize}
\end{frame}

\begin{frame}[fragile]
\frametitle{L-systems example - Sierpinski}
\begin{code}
axiom = "FX+"

production_rules = {
   'X' : 'YF+XF+Y',
   'Y' : 'XF-YF-X'
}
\end{code}

\lstinline|'F'| = move forward one position

\lstinline|'+'| = turn right

\lstinline|'-'| = turn left
\end{frame}

\begin{frame}[fragile]
\frametitle{L-systems example - Sierpinski}
Writing code to produce these using lots of our python course
knowledge: functions, classes, dictionaries,
lists, strings, math, matplotlib.
\end{frame}

\begin{frame}[fragile]
\frametitle{What's wrong?}
\begin{code}
temp = 40 * math.sin(2.5)
if temp < 10
   print("Feeling cold! Only " + str(temp))
   print("wear trousers")
else temp >= 10
   print("We're enjoying summer")
   print("wear shorts")
\end{code}
\end{frame}

\begin{frame}[fragile]
\frametitle{Lists}
\begin{code}
>>> x = [1,3,4,5,7,2]
\end{code}
The last element in x is at what position?

\bigskip

What slice will give me [4,5,7,2] ?

\bigskip

How do I add an item such as 8?
\end{frame}


\begin{frame}[fragile]
\frametitle{Dictionaries}
\begin{code}
my_dict = { "a": 34, "b":28, "c":56 }
\end{code}
How do I add an item "d" with key 77?

\bigskip

How do I check if item "b" is present?

\bigskip

How do I find out the value for item "a"?

\bigskip

How do I get a list of all the keys?
\end{frame}

\begin{frame}[fragile]
\frametitle{Loops}
What does a for loop look like?

What does a while loop look like?

When do you use each one?
\end{frame}

\begin{frame}[fragile]
\frametitle{Functions}
What's the difference between return and print?

Write a function to take a temperature in Celsius and
approximately convert it to Fahrenheit, returning the result. 

(times 1.8 andsubtract 32)
\end{frame}

\begin{frame}[fragile]
\frametitle{Colons and indentation}
When do you use a colon?

When do you need to indent?
\end{frame}

\begin{frame}[fragile]
\frametitle{What does this do?}
\begin{code}
with open("foo.txt","w") as x:
\end{code}

What might you do next?
\end{frame}

\begin{frame}[fragile]
\frametitle{What will happen?}
\begin{code}
>>> 6 / 0
>>> 3 + "hello"
\end{code}

What could you do to deal with problems in:
\begin{itemize}
\item \lstinline{z = x / y}
\item \lstinline{f = open(filename, 'r')}
\end{itemize}
\end{frame}

\begin{frame}[fragile]
\frametitle{What is the role of the following?}
\begin{code}
#
'
"
"""
\end{code}
\end{frame}

\begin{frame}[fragile]
\frametitle{What is happening?}
\begin{code}
import requests

url = 'http://api.yr.no/weatherapi/locationforecast/1.9/?lat=52.41616&lon=-4.064598'
response = requests.get(url)
if response.status_code == 200:
   # What does this check?
else:
   # What would this mean?
\end{code}
\end{frame}


\begin{frame}[fragile]
\frametitle{Summary}
Python the language, and some of its libraries 
\begin{itemize}
\item variables, types, functions
\item data structures (lists, sets, dictionaries, numpy arrays)
\item loops (for, while), if-elif-else
\item modules, exceptions
\item objects and classes
\end{itemize}
\begin{itemize}
\item math, numpy
\item csv
\item requests, BeautifulSoup
\item matplotlib
\end{itemize}
\end{frame}

\end{document}
